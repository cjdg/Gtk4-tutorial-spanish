\hypertarget{string-and-memory-management}{%
\section{String and memory
management}\label{string-and-memory-management}}

GtkTextView and GtkTextBuffer have functions that use string parameters
or return a string. The knowledge of strings and memory management is
useful to understand how to use these functions.

\hypertarget{string-and-memory}{%
\subsection{String and memory}\label{string-and-memory}}

A String is an array of characters that is terminated with
`\textbackslash0'. Strings are not a C type such as char, int, float or
double, but exist as a pointer to a character array. They behaves like a
string type which you may be familiar from other languages. So, this
pointer is often called `a string'.

In the following, \passthrough{\lstinline!a!} and
\passthrough{\lstinline!b!} defined as character arrays, and are
strings.

\begin{lstlisting}[language=C]
char a[10], *b;

a[0] = 'H';
a[1] = 'e';
a[2] = 'l';
a[3] = 'l';
a[4] = 'o';
a[5] = '\0';

b = a;
/* *b is 'H' */
/* *(++b) is 'e' */
\end{lstlisting}

The array \passthrough{\lstinline!a!} has \passthrough{\lstinline!char!}
elements and the size of ten. The first six elements are `H', `e', `l',
`l', `o' and `\textbackslash0'. This array represents the string
``Hello''. The first five elements are character codes that correspond
to the characters. The sixth element is `\textbackslash0', which is the
same as zero, and indicates that the string ends there. The size of the
array is 10, so 4 bytes aren't used, but that's OK, they are just
ignored.

The variable `b' is a pointer to a character. Because
\passthrough{\lstinline!b!} is assigned to be
\passthrough{\lstinline!a!}, \passthrough{\lstinline!a!} and
\passthrough{\lstinline!b!} point the same character (`H'). The variable
\passthrough{\lstinline!a!} is defined as an array and it can't be
changed. It always point the top address of the array. On the other
hand, `b' is a pointer, which is mutable, so \passthrough{\lstinline!b!}
can be change. It is then possible to write statements like
\passthrough{\lstinline!++b!}, which means take the value in b (n
address), increase it by one, and store that back in
\passthrough{\lstinline!b!}.

If a pointer is NULL, it points to nothing. So, the pointer is not a
string. A NULL string on the other hand will be a pointer which points
to a location that contains \passthrough{\lstinline!\\0!}, which is a
string of length 0 (or ""). Programs that use strings will include bugs
if you aren't careful when using NULL pointers.

Another annoying problem is the memory that a string is allocated. There
are four cases:

\begin{itemize}
\tightlist
\item
  The string is read only;
\item
  The string is in static memory area;
\item
  The string is in stack; and
\item
  The string is in memory allocated from the heap area.
\end{itemize}

\hypertarget{read-only-string}{%
\subsection{Read only string}\label{read-only-string}}

A string literal in a C program is surrounded by double quotes and
written as the following:

\begin{lstlisting}[language=C]
char *s;
s = "Hello"
\end{lstlisting}

``Hello'' is a string literal, and is stored in program memory. A string
literal is read only. In the program above, \passthrough{\lstinline!s!}
points the string literal.

So, the following program is illegal.

\begin{lstlisting}[language=C]
*(s+1) = 'a';
\end{lstlisting}

The result is undefined. Probably a bad thing will happen, for example,
a segmentation fault.

NOTE: The memory of the literal string is allocated when the program is
compiled. It is possible to view all the literal strings defined in your
program by using the \passthrough{\lstinline!string!} command.

\hypertarget{strings-defined-as-arrays}{%
\subsection{Strings defined as arrays}\label{strings-defined-as-arrays}}

If a string is defined as an array, it's in either stored in the static
memory area or stack. This depends on the class of the array. If the
array's class is \passthrough{\lstinline!static!}, then it's placed in
static memory area. This allocation and memory address is fixed for the
life of the program. This area can be changed and is writable.

If the array's class is \passthrough{\lstinline!auto!}, then it's placed
in stack. If the array is defined inside a function, its default class
is \passthrough{\lstinline!auto!}. The stack area will disappear when
the function exits and returns to the caller. Arrays defined on the
stack are writable.

\begin{lstlisting}[language=C]
static char a[] = {'H', 'e', 'l', 'l', 'o', '\0'};

void
print_strings (void) {
  char b[] = "Hello";

  a[1] = 'a'; /* Because the array is static, it's writable. */
  b[1] = 'a'; /* Because the array is auto, it's writable. */

  printf ("%s\n", a); /* Hallo */
  printf ("%s\n", b); /* Hallo */
}
\end{lstlisting}

The array \passthrough{\lstinline!a!} is defined externally to a
function and is global in its scope. Such variables are placed in static
memory area even if the \passthrough{\lstinline!static!} class is left
out. The compiler calculates the number of the elements in the right
hand side (six), and then creates code that allocates six bytes in the
static memory area and copies the data to this memory.

The array \passthrough{\lstinline!b!} is defined inside the function so
its class is \passthrough{\lstinline!auto!}. The compiler calculates the
number of the elements in the string literal. It has six elements as the
zero termination character is also included. The compiler creates code
which allocates six bytes memory in the stack and copies the data to the
memory.

Both \passthrough{\lstinline!a!} and \passthrough{\lstinline!b!} are
writable.

The memory is managed by the executable program. You don't need your
program to allocate or free the memory for \passthrough{\lstinline!a!}
and \passthrough{\lstinline!b!}. The array \passthrough{\lstinline!a!}
is created then the program is first run and remains for the life of the
program. The array \passthrough{\lstinline!b!} is created on the stack
then the function is called, disappears when the function returns.

\hypertarget{strings-in-the-heap-area}{%
\subsection{Strings in the heap area}\label{strings-in-the-heap-area}}

You can also get, use and release memory from the heap area. The
standard C library provides \passthrough{\lstinline!malloc!} to get
memory and \passthrough{\lstinline!free!} to put back memory. GLib
provides the functions \passthrough{\lstinline!g\_new!} and
\passthrough{\lstinline!g\_free!} to do the same thing, with support for
some additional Glib functionality.

\begin{lstlisting}[language=C]
g_new (struct_type, n_struct)
\end{lstlisting}

\passthrough{\lstinline!g\_new!} is a macro to allocate memory for an
array.

\begin{itemize}
\tightlist
\item
  \passthrough{\lstinline!struct\_type!} is the type of the element of
  the array.
\item
  \passthrough{\lstinline!n\_struct!} is the size of the array.
\item
  The return value is a pointer to the array. Its type is a pointer to
  \passthrough{\lstinline!struct\_type!}.
\end{itemize}

For example,

\begin{lstlisting}[language=C]
char *s;
s = g_new (char, 10);
/* s points an array of char. The size of the array is 10. */

struct tuple {int x, y;} *t;
t = g_new (struct tuple, 5);
/* t points an array of struct tuple. */
/* The size of the array is 5. */
\end{lstlisting}

\passthrough{\lstinline!g\_free!} frees memory.

\begin{lstlisting}[language=C]
void
g_free (gpointer mem);
\end{lstlisting}

If \passthrough{\lstinline!mem!} is NULL,
\passthrough{\lstinline!g\_free!} does nothing.
\passthrough{\lstinline!gpointer!} is a type of general pointer. It is
the same as \passthrough{\lstinline!void *!}. This pointer can be casted
to any pointer type. Conversely, any pointer type can be casted to
\passthrough{\lstinline!gpointer!}.

\begin{lstlisting}[language=C]
g_free (s);
/* Frees the memory allocated to s. */

g_free (t);
/* Frees the memory allocated to t. */
\end{lstlisting}

If the argument doesn't point allocated memory it will cause an error,
specifically, a segmentation fault.

Some Glib functions allocate memory. For example,
\passthrough{\lstinline!g\_strdup!} allocates memory and copies a string
given as an argument.

\begin{lstlisting}[language=C]
char *s;
s = g_strdup ("Hello");
g_free (s);
\end{lstlisting}

The string literal ``Hello'' has 6 bytes because the string has
`\textbackslash0' at the end it. \passthrough{\lstinline!g\_strdup!}
gets 6 bytes from the heap area and copies the string to the memory.
\passthrough{\lstinline!s!} is assigned the top address of the memory.
\passthrough{\lstinline!g\_free!} returns the memory to the heap area.

\passthrough{\lstinline!g\_strdup!} is described in
\href{https://docs.gtk.org/glib/func.strdup.html}{GLib API Reference}.
The following is extracted from the reference.

\begin{quote}
The returned string should be freed with
\passthrough{\lstinline!g\_free()!} when no longer needed.
\end{quote}

The function reference will describe if the returned value needs to be
freed. If you forget to free the allocated memory it will remain
allocated. Repeated use will cause more memory to be allocated to the
program, which will grow over time. This is called a memory leak, and
the only way to address this bug is to close the program (and restart
it), which will automatically release all of the programs memory back to
the system.

Some GLib functions return a string which mustn't be freed by the
caller.

\begin{lstlisting}[language=C]
const char *
g_quark_to_string (GQuark quark);
\end{lstlisting}

This function returns \passthrough{\lstinline!const char*!} type. The
qualifier \passthrough{\lstinline!const!} means that the returned value
is immutable. The characters pointed by the returned value aren't be
allowed to be changed or freed.

If a variable is qualified with \passthrough{\lstinline!const!}, the
variable can't be assigned except during initialization.

\begin{lstlisting}[language=C]
const int x = 10; /* initialization is OK. */

x = 20; /* This is illegal because x is qualified with const */
\end{lstlisting}
