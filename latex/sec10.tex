\hypertarget{build-system}{%
\section{Build system}\label{build-system}}

\hypertarget{what-do-we-need-to-think-about-to-manage-big-source-files}{%
\subsection{What do we need to think about to manage big source
files?}\label{what-do-we-need-to-think-about-to-manage-big-source-files}}

We've compiled a small editor so far. But Some bad signs are beginning
to appear.

\begin{itemize}
\tightlist
\item
  We've had only one C source file and put everything into it. We need
  to sort it out.
\item
  There are two compilers, \passthrough{\lstinline!gcc!} and
  \passthrough{\lstinline!glib-compile-resources!}. We should control
  them by one building tool.
\end{itemize}

These ideas are useful to manage big source files.

\hypertarget{divide-a-c-source-file-into-two-parts.}{%
\subsection{Divide a C source file into two
parts.}\label{divide-a-c-source-file-into-two-parts.}}

When you divide C source file into several parts, each file should
contain only one thing. For example, our source has two things, the
definition of TfeTextView and functions related to GtkApplication and
GtkApplicationWindow. It is a good idea to separate them into two files,
\passthrough{\lstinline!tfetextview.c!} and
\passthrough{\lstinline!tfe.c!}.

\begin{itemize}
\tightlist
\item
  \passthrough{\lstinline!tfetextview.c!} includes the definition and
  functions of TfeTextView.
\item
  \passthrough{\lstinline!tfe.c!} includes functions like
  \passthrough{\lstinline!main!},
  \passthrough{\lstinline!app\_activate!},
  \passthrough{\lstinline!app\_open!} and so on, which relate to
  GtkApplication and GtkApplicationWindow
\end{itemize}

Now we have three source files, \passthrough{\lstinline!tfetextview.c!},
\passthrough{\lstinline!tfe.c!} and \passthrough{\lstinline!tfe3.ui!}.
The \passthrough{\lstinline!3!} of \passthrough{\lstinline!tfe3.ui!} is
like a version number. Managing version with filenames is one possible
idea but it may make bothersome problem. You need to rewrite filename in
each version and it affects to contents of source files that refer to
filenames. So, we should take \passthrough{\lstinline!3!} away from the
filename.

In \passthrough{\lstinline!tfe.c!} the function
\passthrough{\lstinline!tfe\_text\_view\_new!} is invoked to create a
TfeTextView instance. But it is defined in
\passthrough{\lstinline!tfetextview.c!}, not
\passthrough{\lstinline!tfe.c!}. The lack of the declaration (not
definition) of \passthrough{\lstinline!tfe\_text\_view\_new!} makes
error when \passthrough{\lstinline!tfe.c!} is compiled. The declaration
is necessary in \passthrough{\lstinline!tfe.c!}. Those public
information is usually written in header files. It has
\passthrough{\lstinline!.h!} suffix like
\passthrough{\lstinline!tfetextview.h!} And header files are included by
C source files. For example, \passthrough{\lstinline!tfetextview.h!} is
included by \passthrough{\lstinline!tfe.c!}.

All the source files are listed below.

\passthrough{\lstinline!tfetextview.h!}

\begin{lstlisting}[language=C, numbers=left]
#include <gtk/gtk.h>

#define TFE_TYPE_TEXT_VIEW tfe_text_view_get_type ()
G_DECLARE_FINAL_TYPE (TfeTextView, tfe_text_view, TFE, TEXT_VIEW, GtkTextView)

void
tfe_text_view_set_file (TfeTextView *tv, GFile *f);

GFile *
tfe_text_view_get_file (TfeTextView *tv);

GtkWidget *
tfe_text_view_new (void);
\end{lstlisting}

\passthrough{\lstinline!tfetextview.c!}

\begin{lstlisting}[language=C, numbers=left]
#include <gtk/gtk.h>
#include "tfetextview.h"

struct _TfeTextView
{
  GtkTextView parent;
  GFile *file;
};

G_DEFINE_TYPE (TfeTextView, tfe_text_view, GTK_TYPE_TEXT_VIEW);

static void
tfe_text_view_init (TfeTextView *tv) {
}

static void
tfe_text_view_class_init (TfeTextViewClass *class) {
}

void
tfe_text_view_set_file (TfeTextView *tv, GFile *f) {
  tv -> file = f;
}

GFile *
tfe_text_view_get_file (TfeTextView *tv) {
  return tv -> file;
}

GtkWidget *
tfe_text_view_new (void) {
  return GTK_WIDGET (g_object_new (TFE_TYPE_TEXT_VIEW, NULL));
}
\end{lstlisting}

\passthrough{\lstinline!tfe.c!}

\begin{lstlisting}[language=C, numbers=left]
#include <gtk/gtk.h>
#include "tfetextview.h"

static void
app_activate (GApplication *app, gpointer user_data) {
  g_print ("You need a filename argument.\n");
}

static void
app_open (GApplication *app, GFile ** files, gint n_files, gchar *hint, gpointer user_data) {
  GtkWidget *win;
  GtkWidget *nb;
  GtkWidget *lab;
  GtkNotebookPage *nbp;
  GtkWidget *scr;
  GtkWidget *tv;
  GtkTextBuffer *tb;
  char *contents;
  gsize length;
  char *filename;
  int i;
  GtkBuilder *build;

  build = gtk_builder_new_from_resource ("/com/github/ToshioCP/tfe3/tfe.ui");
  win = GTK_WIDGET (gtk_builder_get_object (build, "win"));
  gtk_window_set_application (GTK_WINDOW (win), GTK_APPLICATION (app));
  nb = GTK_WIDGET (gtk_builder_get_object (build, "nb"));
  g_object_unref (build);
  for (i = 0; i < n_files; i++) {
    if (g_file_load_contents (files[i], NULL, &contents, &length, NULL, NULL)) {
      scr = gtk_scrolled_window_new ();
      tv = tfe_text_view_new ();
      tb = gtk_text_view_get_buffer (GTK_TEXT_VIEW (tv));
      gtk_text_view_set_wrap_mode (GTK_TEXT_VIEW (tv), GTK_WRAP_WORD_CHAR);
      gtk_scrolled_window_set_child (GTK_SCROLLED_WINDOW (scr), tv);

      tfe_text_view_set_file (TFE_TEXT_VIEW (tv),  g_file_dup (files[i]));
      gtk_text_buffer_set_text (tb, contents, length);
      g_free (contents);
      filename = g_file_get_basename (files[i]);
      lab = gtk_label_new (filename);
      gtk_notebook_append_page (GTK_NOTEBOOK (nb), scr, lab);
      nbp = gtk_notebook_get_page (GTK_NOTEBOOK (nb), scr);
      g_object_set (nbp, "tab-expand", TRUE, NULL);
      g_free (filename);
    } else if ((filename = g_file_get_path (files[i])) != NULL) {
        g_print ("No such file: %s.\n", filename);
        g_free (filename);
    } else
        g_print ("No valid file is given\n");
  }
  if (gtk_notebook_get_n_pages (GTK_NOTEBOOK (nb)) > 0) {
    gtk_widget_show (win);
  } else
    gtk_window_destroy (GTK_WINDOW (win));
}

int
main (int argc, char **argv) {
  GtkApplication *app;
  int stat;

  app = gtk_application_new ("com.github.ToshioCP.tfe", G_APPLICATION_HANDLES_OPEN);
  g_signal_connect (app, "activate", G_CALLBACK (app_activate), NULL);
  g_signal_connect (app, "open", G_CALLBACK (app_open), NULL);
  stat =g_application_run (G_APPLICATION (app), argc, argv);
  g_object_unref (app);
  return stat;
}
\end{lstlisting}

The ui file \passthrough{\lstinline!tfe.ui!} is the same as
\passthrough{\lstinline!tfe3.ui!} in the previous section.

\passthrough{\lstinline!tfe.gresource.xml!}

\begin{lstlisting}[language=XML, numbers=left]
<?xml version="1.0" encoding="UTF-8"?>
<gresources>
  <gresource prefix="/com/github/ToshioCP/tfe3">
    <file>tfe.ui</file>
  </gresource>
</gresources>
\end{lstlisting}

\hypertarget{make}{%
\subsection{Make}\label{make}}

Dividing a file makes it easy to maintain source files. But now we are
faced with a new problem. The building step increases.

\begin{itemize}
\tightlist
\item
  Compiling the ui file \passthrough{\lstinline!tfe.ui!} into
  \passthrough{\lstinline!resources.c!}.
\item
  Compiling \passthrough{\lstinline!tfe.c!} into
  \passthrough{\lstinline!tfe.o!} (object file).
\item
  Compiling \passthrough{\lstinline!tfetextview.c!} into
  \passthrough{\lstinline!tfetextview.o!}.
\item
  Compiling \passthrough{\lstinline!resources.c!} into
  \passthrough{\lstinline!resources.o!}.
\item
  Linking all the object files into application
  \passthrough{\lstinline!tfe!}.
\end{itemize}

Now build tool is necessary to manage it. Make is one of the build
tools. It was created in 1976. It is an old and widely used program.

Make analyzes Makefile and executes compilers. All instructions are
written in Makefile.

\begin{lstlisting}[language=make]
sample.o: sample.c
    gcc -o sample.o sample.c
\end{lstlisting}

The sample of Malefile above consists of three elements,
\passthrough{\lstinline!sample.o!}, \passthrough{\lstinline!sample.c!}
and \passthrough{\lstinline!gcc -o sample.o sample.c!}.

\begin{itemize}
\tightlist
\item
  \passthrough{\lstinline!sample.o!} is called target.
\item
  \passthrough{\lstinline!sample.c!} is prerequisite.
\item
  \passthrough{\lstinline!gcc -o sample.o sample.c!} is recipe. Recipes
  follow tab characters, not spaces. (It is very important. Use tab not
  space, or make won't work as you expected).
\end{itemize}

The rule is:

If a prerequisite modified later than a target, then make executes the
recipe.

In the example above, if \passthrough{\lstinline!sample.c!} is modified
after the generation of \passthrough{\lstinline!sample.o!}, then make
executes gcc and compile \passthrough{\lstinline!sample.c!} into
\passthrough{\lstinline!sample.o!}. If the modification time of
\passthrough{\lstinline!sample.c!} is older then the generation of
\passthrough{\lstinline!sample.o!}, then no compiling is necessary, so
make does nothing.

The Makefile for \passthrough{\lstinline!tfe!} is as follows.

\begin{lstlisting}[language=make, numbers=left]
all: tfe

tfe: tfe.o tfetextview.o resources.o
    gcc -o tfe tfe.o tfetextview.o resources.o `pkg-config --libs gtk4`

tfe.o: tfe.c tfetextview.h
    gcc -c -o tfe.o `pkg-config --cflags gtk4` tfe.c
tfetextview.o: tfetextview.c tfetextview.h
    gcc -c -o tfetextview.o `pkg-config --cflags gtk4` tfetextview.c
resources.o: resources.c
    gcc -c -o resources.o `pkg-config --cflags gtk4` resources.c

resources.c: tfe.gresource.xml tfe.ui
    glib-compile-resources tfe.gresource.xml --target=resources.c --generate-source

.Phony: clean

clean:
    rm -f tfe tfe.o tfetextview.o resources.o resources.c
\end{lstlisting}

You only need to type \passthrough{\lstinline!make!}.

\begin{lstlisting}
$ make
gcc -c -o tfe.o `pkg-config --cflags gtk4` tfe.c
gcc -c -o tfetextview.o `pkg-config --cflags gtk4` tfetextview.c
glib-compile-resources tfe.gresource.xml --target=resources.c --generate-source
gcc -c -o resources.o `pkg-config --cflags gtk4` resources.c
gcc -o tfe tfe.o tfetextview.o resources.o `pkg-config --libs gtk4`
\end{lstlisting}

I used only very basic rules to write this Makefile. There are many more
convenient methods to make it more compact. But it will be long to
explain it. So I want to finish explaining make and move on to the next
topic.

\hypertarget{rake}{%
\subsection{Rake}\label{rake}}

Rake is a similar program to make. It is written in Ruby code. If you
don't use Ruby, you don't need to read this subsection. However, Ruby is
really sophisticated and recommendable script language.

\begin{itemize}
\tightlist
\item
  Rakefile controls the behavior of \passthrough{\lstinline!rake!}.
\item
  You can write any Ruby code in Rakefile.
\end{itemize}

Rake has task and file task, which is similar to target, prerequisite
and recipe in make.

\begin{lstlisting}[language=Ruby, numbers=left]
require 'rake/clean'

targetfile = "tfe"
srcfiles = FileList["tfe.c", "tfetextview.c", "resources.c"]
rscfile = srcfiles[2]
objfiles = srcfiles.gsub(/.c$/, '.o')

CLEAN.include(targetfile, objfiles, rscfile)

task default: targetfile

file targetfile => objfiles do |t|
  sh "gcc -o #{t.name} #{t.prerequisites.join(' ')} `pkg-config --libs gtk4`"
end

objfiles.each do |obj|
  src = obj.gsub(/.o$/,'.c')
  file obj => src do |t|
    sh "gcc -c -o #{t.name} `pkg-config --cflags gtk4` #{t.source}"
  end
end

file rscfile => ["tfe.gresource.xml", "tfe.ui"] do |t|
  sh "glib-compile-resources #{t.prerequisites[0]} --target=#{t.name} --generate-source"
end
\end{lstlisting}

The contents of the \passthrough{\lstinline!Rakefile!} is almost same as
the \passthrough{\lstinline!Makefile!} in the previous subsection.

\begin{itemize}
\tightlist
\item
  3-6: Defines target file, source file and so on.
\item
  1, 8: Loads clean library. And defines CLEAN file list. The files
  included by CLEAN will be removed when
  \passthrough{\lstinline!rake clean!} is typed on the command line.
\item
  10: The default target depends on
  \passthrough{\lstinline!targetfile!}. The task
  \passthrough{\lstinline!default!} is the final goal of tasks.
\item
  12-14: \passthrough{\lstinline!targetfile!} depends on
  \passthrough{\lstinline!objfiles!}. The variable
  \passthrough{\lstinline!t!} is a task object.

  \begin{itemize}
  \tightlist
  \item
    \passthrough{\lstinline!t.name!} is a target name
  \item
    \passthrough{\lstinline!t.prerequisites!} is an array of
    prerequisites.
  \item
    \passthrough{\lstinline!t.source!} is the first element of
    prerequisites.
  \end{itemize}
\item
  \passthrough{\lstinline!sh!} is a method to give the following string
  to shell as an argument and executes the shell.
\item
  16-21: There's a loop by each element of the array of
  \passthrough{\lstinline!objfiles!}. Each object depends on
  corresponding source file.
\item
  23-25: Resource file depends on xml file and ui file.
\end{itemize}

Rakefile might seem to be difficult for beginners. But, you can use any
Ruby syntax in Rakefile, so it is really flexible. If you practice Ruby
and Rakefile, it will be highly productive tools.

\hypertarget{meson-and-ninja}{%
\subsection{Meson and ninja}\label{meson-and-ninja}}

Meson is one of the most popular building tool despite the developing
version. And ninja is similar to make but much faster than make. Several
years ago, most of the C developers used autotools and make. But now the
situation has changed. Many developers are using meson and ninja now.

To use meson, you first need to write
\passthrough{\lstinline!meson.build!} file.

\begin{lstlisting}[numbers=left]
project('tfe', 'c')

gtkdep = dependency('gtk4')

gnome=import('gnome')
resources = gnome.compile_resources('resources','tfe.gresource.xml')

sourcefiles=files('tfe.c', 'tfetextview.c')

executable('tfe', sourcefiles, resources, dependencies: gtkdep)
\end{lstlisting}

\begin{itemize}
\tightlist
\item
  1: The function \passthrough{\lstinline!project!} defines things about
  the project. The first parameter is the name of the project and the
  second is the programming language.
\item
  2: \passthrough{\lstinline!dependency!} function defines a dependency
  that is taken by \passthrough{\lstinline!pkg-config!}. We put
  \passthrough{\lstinline!gtk4!} as an argument.
\item
  5: \passthrough{\lstinline!import!} function imports a module. In line
  5, the gnome module is imported and assigned to the variable
  \passthrough{\lstinline!gnome!}. The gnome module provides helper
  tools to build GTK programs.
\item
  6: \passthrough{\lstinline!.compile\_resources!} is a method of the
  gnome module and compiles files to resources under the instruction of
  xml file. In line 6, the resource filename is
  \passthrough{\lstinline!resources!}, which means
  \passthrough{\lstinline!resources.c!} and
  \passthrough{\lstinline!resources.h!}, and xml file is
  \passthrough{\lstinline!tfe.gresource.xml!}. This method generates C
  source file by default.
\item
  8: Defines source files.
\item
  10: Executable function generates a target file by compiling source
  files. The first parameter is the filename of the target. The
  following parameters are source files. The last parameter is an option
  \passthrough{\lstinline!dependencies!}.
  \passthrough{\lstinline!gtkdep!} is used in the compilation.
\end{itemize}

Now run meson and ninja.

\begin{lstlisting}
$ meson _build
$ ninja -C _build
\end{lstlisting}

Then, the executable file \passthrough{\lstinline!tfe!} is generated
under the directory \passthrough{\lstinline!\_build!}.

\begin{lstlisting}
$ _build/tfe tfe.c tfetextview.c
\end{lstlisting}

Then the window appears. And two notebook pages are in the window. One
notebook is \passthrough{\lstinline!tfe.c!} and the other is
\passthrough{\lstinline!tfetextview.c!}.

I've shown you three build tools. I think meson and ninja is the best
choice for the present.

We divided a file into some categorized files and used a build tool.
This method is used by many developers.
