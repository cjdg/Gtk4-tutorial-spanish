\hypertarget{template-xml-and-composite-widget}{%
\section{Template XML and composite
widget}\label{template-xml-and-composite-widget}}

The tfe program in the previous section is not so good because many
things are crammed into \passthrough{\lstinline!tfepplication.c!}. Many
static variables in \passthrough{\lstinline!tfepplication.c!} shows
that.

\begin{lstlisting}[language=C]
static GtkDialog *pref;
static GtkFontButton *fontbtn;
static GSettings *settings;
static GtkDialog *alert;
static GtkLabel *lb_alert;
static GtkButton *btn_accept;

static gulong pref_close_request_handler_id = 0;
static gulong alert_close_request_handler_id = 0;
static gboolean is_quit;
\end{lstlisting}

Generally, if there are many global or static variables in the program,
it is not a good program. Such programs are difficult to maintain.

The file \passthrough{\lstinline!tfeapplication.c!} should be divided
into several files.

\begin{itemize}
\tightlist
\item
  \passthrough{\lstinline!tfeapplication.c!} only has codes related to
  GtkApplication.
\item
  A file for GtkApplicationWindow
\item
  A file for a preference dialog
\item
  A file for an alert dialog
\end{itemize}

The preference dialog is defined by a ui file. And it has GtkBox,
GtkLabel and GtkFontButton in it. Such widget is called composite
widget. Composite widget is a child object (not child widget) of a
widget. For example, the preference composite widget is a child object
of GtkDialog. Composite widget can be built from template XML. Next
subsection shows how to build a preference dialog.

\hypertarget{preference-dialog}{%
\subsection{Preference dialog}\label{preference-dialog}}

First, write a template XML file.

\begin{lstlisting}[language=XML, numbers=left]
<?xml version="1.0" encoding="UTF-8"?>
<interface>
  <template class="TfePref" parent="GtkDialog">
    <property name="title">Preferences</property>
    <property name="resizable">FALSE</property>
    <property name="modal">TRUE</property>
    <child internal-child="content_area">
      <object class="GtkBox" id="content_area">
        <child>
          <object class="GtkBox" id="pref_boxh">
            <property name="orientation">GTK_ORIENTATION_HORIZONTAL</property>
            <property name="spacing">12</property>
            <property name="margin-start">12</property>
            <property name="margin-end">12</property>
            <property name="margin-top">12</property>
            <property name="margin-bottom">12</property>
            <child>
              <object class="GtkLabel" id="fontlabel">
                <property name="label">Font:</property>
                <property name="xalign">1</property>
              </object>
            </child>
            <child>
              <object class="GtkFontButton" id="fontbtn">
              </object>
            </child>
          </object>
        </child>
      </object>
    </child>
  </template>
</interface>
\end{lstlisting}

\begin{itemize}
\tightlist
\item
  3: Template tag specifies a composite widget. The value of a class
  attribute is the object name of the composite widget. This XML file
  names the object ``TfePref''. It is defined in a C source file and it
  will be shown later. A parent attribute specifies the direct parent
  object of the composite widget. \passthrough{\lstinline!TfePref!} is a
  child object of \passthrough{\lstinline!GtkDialog!}. Therefore the
  value of the attribute is ``GtkDialog''. A parent attribute is
  optional but it is recommended to specify.
\end{itemize}

Other lines are the same as before. The object
\passthrough{\lstinline!TfePref!} is defined in
\passthrough{\lstinline!tfepref.h!} and
\passthrough{\lstinline!tfepref.c!}.

\begin{lstlisting}[language=C, numbers=left]
#ifndef __TFE_PREF_H__
#define __TFE_PREF_H__

#include <gtk/gtk.h>

#define TFE_TYPE_PREF tfe_pref_get_type ()
G_DECLARE_FINAL_TYPE (TfePref, tfe_pref, TFE, PREF, GtkDialog)

GtkWidget *
tfe_pref_new (GtkWindow *win);

#endif /* __TFE_PREF_H__ */
\end{lstlisting}

\begin{itemize}
\tightlist
\item
  6-7: When you define a new object, you need to write these two lines.
  Refer to Section 8.
\item
  9-10: \passthrough{\lstinline!tfe\_pref\_new!} creates a new TfePref
  object. It has a parameter \passthrough{\lstinline!win!} which is used
  as a transient parent window to show the dialog.
\end{itemize}

\begin{lstlisting}[language=C, numbers=left]
#include "tfepref.h"

struct _TfePref
{
  GtkDialog parent;
  GSettings *settings;
  GtkFontButton *fontbtn;
};

G_DEFINE_TYPE (TfePref, tfe_pref, GTK_TYPE_DIALOG);

static void
tfe_pref_dispose (GObject *gobject) {
  TfePref *pref = TFE_PREF (gobject);

  g_clear_object (&pref->settings);
  G_OBJECT_CLASS (tfe_pref_parent_class)->dispose (gobject);
}

static void
tfe_pref_init (TfePref *pref) {
  gtk_widget_init_template (GTK_WIDGET (pref));
  pref->settings = g_settings_new ("com.github.ToshioCP.tfe");
  g_settings_bind (pref->settings, "font", pref->fontbtn, "font", G_SETTINGS_BIND_DEFAULT);
}

static void
tfe_pref_class_init (TfePrefClass *class) {
  GObjectClass *object_class = G_OBJECT_CLASS (class);

  object_class->dispose = tfe_pref_dispose;
  gtk_widget_class_set_template_from_resource (GTK_WIDGET_CLASS (class), "/com/github/ToshioCP/tfe/tfepref.ui");
  gtk_widget_class_bind_template_child (GTK_WIDGET_CLASS (class), TfePref, fontbtn);
}

GtkWidget *
tfe_pref_new (GtkWindow *win) {
  return GTK_WIDGET (g_object_new (TFE_TYPE_PREF, "transient-for", win, NULL));
}
\end{lstlisting}

\begin{itemize}
\tightlist
\item
  3-8: The structure of an instance of this object. It has two
  variables, settings and fontbtn.
\item
  10: \passthrough{\lstinline!G\_DEFINE\_TYPE!} macro. This macro
  registers the TfePref type.
\item
  12-18: Dispose handler. This handler is called when the instance is
  destroyed. The destruction process has two stages, disposing and
  finalizing. When disposing, the instance releases all the references
  (to the other instances). TfePref object holds a reference to the
  GSettings instance. It is released in line 16. After that parents
  dispose handler is called in line 17. For further information about
  destruction process, refer to Section 11.
\item
  27-34: Class initialization function.
\item
  31: Set the dispose handler.
\item
  32:
  \passthrough{\lstinline!gtk\_widget\_class\_set\_template\_from\_resource!}
  function associates the description in the XML file
  (\passthrough{\lstinline!tfepref.ui!}) with the widget. At this moment
  no instance is created. It just make the class to know the structure
  of the object. That's why the top level tag is not
  \passthrough{\lstinline!<object>!} but
  \passthrough{\lstinline!<template>!} in the XML file.
\item
  33:
  \passthrough{\lstinline!gtk\_widget\_class\_bind\_template\_child!}
  function binds a private variable of the object with a child object in
  the template. This function is a macro. The name of the private
  variable (\passthrough{\lstinline!fontbtn!} in line 7) and the id
  \passthrough{\lstinline!fontbtn!} in the XML file (line 24) must be
  the same. The pointer to the instance will be assigned to the variable
  \passthrough{\lstinline!fontbtn!} when the instance is created.
\item
  20-25: Instance initialization function.
\item
  22: Creates the instance based on the template in the class. The
  template has been made during the class initialization process.
\item
  23: Create GSettings instance with the id ``com.github.ToshioCP.tfe''.
\item
  24: Bind the font key in the GSettings object to the font property in
  the GtkFontButton.
\item
  36-39: The function \passthrough{\lstinline!tfe\_pref\_new!} creates
  an instance of TfePref. The parameter \passthrough{\lstinline!win!} is
  a transient parent.
\end{itemize}

Now, It is very simple to use this dialog. A caller just creates this
object and shows it.

\begin{lstlisting}[language=C]
TfePref *pref;
pref = tfe_pref_new (win) /* win is the top-level window */
gtk_widget_show (GTK_WINDOW (win));
\end{lstlisting}

This instance is automatically destroyed when a user clicks on the close
button. That's all. If you want to show the dialog again, just create
and show it.

\hypertarget{alert-dialog}{%
\subsection{Alert dialog}\label{alert-dialog}}

It is almost same as preference dialog.

Its XML file is:

\begin{lstlisting}[language=XML, numbers=left]
<?xml version="1.0" encoding="UTF-8"?>
<interface>
  <template class="TfeAlert" parent="GtkDialog">
    <property name="title">Are you sure?</property>
    <property name="resizable">FALSE</property>
    <property name="modal">TRUE</property>
    <child internal-child="content_area">
      <object class="GtkBox">
        <child>
          <object class="GtkBox">
            <property name="orientation">GTK_ORIENTATION_HORIZONTAL</property>
            <property name="spacing">12</property>
            <property name="margin-start">12</property>
            <property name="margin-end">12</property>
            <property name="margin-top">12</property>
            <property name="margin-bottom">12</property>
            <child>
              <object class="GtkImage">
                <property name="icon-name">dialog-warning</property>
                <property name="icon-size">GTK_ICON_SIZE_LARGE</property>
              </object>
            </child>
            <child>
              <object class="GtkLabel" id="lb_alert">
              </object>
            </child>
          </object>
        </child>
      </object>
    </child>
    <child type="action">
      <object class="GtkButton" id="btn_cancel">
        <property name="label">Cancel</property>
      </object>
    </child>
    <child type="action">
      <object class="GtkButton" id="btn_accept">
        <property name="label">Close</property>
      </object>
    </child>
    <action-widgets>
      <action-widget response="cancel" default="true">btn_cancel</action-widget>
      <action-widget response="accept">btn_accept</action-widget>
    </action-widgets>
  </template>
</interface>
\end{lstlisting}

The header file is:

\begin{lstlisting}[language=C, numbers=left]
#ifndef __TFE_ALERT_H__
#define __TFE_ALERT_H__

#include <gtk/gtk.h>

#define TFE_TYPE_ALERT tfe_alert_get_type ()
G_DECLARE_FINAL_TYPE (TfeAlert, tfe_alert, TFE, ALERT, GtkDialog)

void
tfe_alert_set_message (TfeAlert *alert, const char *message);

void
tfe_alert_set_button_label (TfeAlert *alert, const char *label);

GtkWidget *
tfe_alert_new (GtkWindow *win);

#endif /* __TFE_ALERT_H__ */
\end{lstlisting}

There are three public functions. The functions
\passthrough{\lstinline!tfe\_alert\_set\_message!} and
\passthrough{\lstinline!tfe\_alert\_set\_button\_label!} sets the label
and button name of the alert dialog. For example, if you want to show an
alert that the user tries to close without saving the content, set them
like:

\begin{lstlisting}[language=C]
tfe_alert_set_message (alert, "Are you really close without saving?"); /* alert points to a TfeAlert instance */
tfe_alert_set_button_label (alert, "Close");
\end{lstlisting}

The function \passthrough{\lstinline!tfe\_alert\_new!} creates a
TfeAlert dialog.

The C source file is:

\begin{lstlisting}[language=C, numbers=left]
#include "tfealert.h"

struct _TfeAlert
{
  GtkDialog parent;
  GtkLabel *lb_alert;
  GtkButton *btn_accept;
};

G_DEFINE_TYPE (TfeAlert, tfe_alert, GTK_TYPE_DIALOG);

void
tfe_alert_set_message (TfeAlert *alert, const char *message) {
  gtk_label_set_text (alert->lb_alert, message);
}

void
tfe_alert_set_button_label (TfeAlert *alert, const char *label) {
  gtk_button_set_label (alert->btn_accept, label);
}

static void
tfe_alert_init (TfeAlert *alert) {
  gtk_widget_init_template (GTK_WIDGET (alert));
}

static void
tfe_alert_class_init (TfeAlertClass *class) {
  gtk_widget_class_set_template_from_resource (GTK_WIDGET_CLASS (class), "/com/github/ToshioCP/tfe/tfealert.ui");
  gtk_widget_class_bind_template_child (GTK_WIDGET_CLASS (class), TfeAlert, lb_alert);
  gtk_widget_class_bind_template_child (GTK_WIDGET_CLASS (class), TfeAlert, btn_accept);
}

GtkWidget *
tfe_alert_new (GtkWindow *win) {
  return GTK_WIDGET (g_object_new (TFE_TYPE_ALERT, "transient-for", win, NULL));
}
\end{lstlisting}

The program is almost same as \passthrough{\lstinline!tfepref.c!}.

The instruction how to use this object is as follows.

\begin{enumerate}
\def\labelenumi{\arabic{enumi}.}
\tightlist
\item
  Write a ``response'' signal handler.
\item
  Create a TfeAlert object.
\item
  Connect ``response'' signal to a handler
\item
  Show the dialog
\item
  In the signal handler, do something with regard to the response-id.
  Then destroy the dialog.
\end{enumerate}

\hypertarget{top-level-window}{%
\subsection{Top-level window}\label{top-level-window}}

In the same way, create a child object of GtkApplicationWindow. The
object name is ``TfeWindow''.

\begin{lstlisting}[language=XML, numbers=left]
<?xml version="1.0" encoding="UTF-8"?>
<interface>
  <template class="TfeWindow" parent="GtkApplicationWindow">
    <property name="title">file editor</property>
    <property name="default-width">600</property>
    <property name="default-height">400</property>
    <child>
      <object class="GtkBox" id="boxv">
        <property name="orientation">GTK_ORIENTATION_VERTICAL</property>
        <child>
          <object class="GtkBox" id="boxh">
            <property name="orientation">GTK_ORIENTATION_HORIZONTAL</property>
            <child>
              <object class="GtkLabel" id="dmy1">
                <property name="width-chars">10</property>
              </object>
            </child>
            <child>
              <object class="GtkButton" id="btno">
                <property name="label">Open</property>
                <property name="action-name">win.open</property>
              </object>
            </child>
            <child>
              <object class="GtkButton" id="btns">
                <property name="label">Save</property>
                <property name="action-name">win.save</property>
              </object>
            </child>
            <child>
              <object class="GtkLabel" id="dmy2">
                <property name="hexpand">TRUE</property>
              </object>
            </child>
            <child>
              <object class="GtkButton" id="btnc">
                <property name="label">Close</property>
                <property name="action-name">win.close</property>
              </object>
            </child>
            <child>
              <object class="GtkMenuButton" id="btnm">
                <property name="direction">down</property>
                <property name="halign">start</property>
                <property name="icon-name">open-menu-symbolic</property>
              </object>
            </child>
            <child>
              <object class="GtkLabel" id="dmy3">
                <property name="width-chars">10</property>
              </object>
            </child>
          </object>
        </child>
        <child>
          <object class="GtkNotebook" id="nb">
            <property name="scrollable">TRUE</property>
            <property name="hexpand">TRUE</property>
            <property name="vexpand">TRUE</property>
          </object>
        </child>
      </object>
    </child>
  </template>
</interface>
\end{lstlisting}

This XML file is almost same as before except template tag and
``action-name'' property in buttons.

GtkButton implements GtkActionable interface, which has ``action-name''
property. If this property is set, GtkButton activates the action when
it is clicked. For example, if an open button is clicked, ``win.open''
action will be activated and \passthrough{\lstinline!open\_activated!}
handler will be invoked.

This action is also used by ``\textless Control\textgreater o''
accelerator (See the source code of
\passthrough{\lstinline!tfewindow.c!} below). If you use ``clicked''
signal for the button, you need its signal handler. Then, there are two
handlers:

\begin{itemize}
\tightlist
\item
  a handler for the ``clicked'' signal on the button
\item
  a handler for the ``activate'' signal on the ``win.open'' action, to
  which ``\textless Control\textgreater o'' accelerator is connected
\end{itemize}

These two handlers do almost same thing. It is inefficient. Connecting
buttons to actions is a good way to reduce unnecessary codes.

\begin{lstlisting}[language=C, numbers=left]
#ifndef __TFE_WINDOW_H__
#define __TFE_WINDOW_H__

#include <gtk/gtk.h>

#define TFE_TYPE_WINDOW tfe_window_get_type ()
G_DECLARE_FINAL_TYPE (TfeWindow, tfe_window, TFE, WINDOW, GtkApplicationWindow)

void
tfe_window_notebook_page_new (TfeWindow *win);

void
tfe_window_notebook_page_new_with_files (TfeWindow *win, GFile **files, int n_files);

GtkWidget *
tfe_window_new (GtkApplication *app);

#endif /* __TFE_WINDOW_H__ */
\end{lstlisting}

There are three public functions. The function
\passthrough{\lstinline!tfe\_window\_notebook\_page\_new!} creates a new
notebook page. This is a wrapper function for
\passthrough{\lstinline!notebook\_page\_new!}. It is called by
GtkApplication object. The function
\passthrough{\lstinline!tfe\_window\_notebook\_page\_new\_with\_files!}
creates notebook pages with a contents read from the given files. The
function \passthrough{\lstinline!tfe\_window\_new!} creates a TfeWindow
instance.

\begin{lstlisting}[language=C, numbers=left]
#include "tfewindow.h"
#include "tfenotebook.h"
#include "tfepref.h"
#include "tfealert.h"
#include "css.h"

struct _TfeWindow {
  GtkApplicationWindow parent;
  GtkMenuButton *btnm;
  GtkNotebook *nb;
  GSettings *settings;
  gboolean is_quit;
};

G_DEFINE_TYPE (TfeWindow, tfe_window, GTK_TYPE_APPLICATION_WINDOW);

/* alert response signal handler */
static void
alert_response_cb (GtkDialog *alert, int response_id, gpointer user_data) {
  TfeWindow *win = TFE_WINDOW (user_data);

  if (response_id == GTK_RESPONSE_ACCEPT) {
    if (win->is_quit)
      gtk_window_destroy(GTK_WINDOW (win));
    else
      notebook_page_close (win->nb);
  }
  gtk_window_destroy (GTK_WINDOW (alert));
}

/* ----- action activated handlers ----- */
static void
open_activated (GSimpleAction *action, GVariant *parameter, gpointer user_data) {
  TfeWindow *win = TFE_WINDOW (user_data);

  notebook_page_open (GTK_NOTEBOOK (win->nb));
}

static void
save_activated (GSimpleAction *action, GVariant *parameter, gpointer user_data) {
  TfeWindow *win = TFE_WINDOW (user_data);

  notebook_page_save (GTK_NOTEBOOK (win->nb));
}

static void
close_activated (GSimpleAction *action, GVariant *parameter, gpointer user_data) {
  TfeWindow *win = TFE_WINDOW (user_data);
  TfeAlert *alert;

  if (has_saved (win->nb))
    notebook_page_close (win->nb);
  else {
    win->is_quit = false;
    alert = TFE_ALERT (tfe_alert_new (GTK_WINDOW (win)));
    tfe_alert_set_message (alert, "Contents aren't saved yet.\nAre you sure to close?");
    tfe_alert_set_button_label (alert, "Close");
    g_signal_connect (GTK_DIALOG (alert), "response", G_CALLBACK (alert_response_cb), win);
    gtk_widget_show (GTK_WIDGET (alert));
  }
}

static void
new_activated (GSimpleAction *action, GVariant *parameter, gpointer user_data) {
  TfeWindow *win = TFE_WINDOW (user_data);

  notebook_page_new (GTK_NOTEBOOK (win->nb));
}

static void
saveas_activated (GSimpleAction *action, GVariant *parameter, gpointer user_data) {
  TfeWindow *win = TFE_WINDOW (user_data);

  notebook_page_saveas (GTK_NOTEBOOK (win->nb));
}

static void
pref_activated (GSimpleAction *action, GVariant *parameter, gpointer user_data) {
  TfeWindow *win = TFE_WINDOW (user_data);
  GtkWidget *pref;

  pref = tfe_pref_new (GTK_WINDOW (win));
  gtk_widget_show (pref);
}

static void
quit_activated (GSimpleAction *action, GVariant *parameter, gpointer user_data) {
  TfeWindow *win = TFE_WINDOW (user_data);

  TfeAlert *alert;

  if (has_saved_all (GTK_NOTEBOOK (win->nb)))
    gtk_window_destroy (GTK_WINDOW (win));
  else {
    win->is_quit = true;
    alert = TFE_ALERT (tfe_alert_new (GTK_WINDOW (win)));
    tfe_alert_set_message (alert, "Contents aren't saved yet.\nAre you sure to quit?");
    tfe_alert_set_button_label (alert, "Quit");
    g_signal_connect (GTK_DIALOG (alert), "response", G_CALLBACK (alert_response_cb), win);
    gtk_widget_show (GTK_WIDGET (alert));
  }
}

/* gsettings changed::font signal handler */
static void
changed_font_cb (GSettings *settings, char *key, gpointer user_data) {
  GtkWindow *win = GTK_WINDOW (user_data); 
  char *font;
  PangoFontDescription *pango_font_desc;

  font = g_settings_get_string (settings, "font");
  pango_font_desc = pango_font_description_from_string (font);
  g_free (font);
  set_font_for_display_with_pango_font_desc (win, pango_font_desc);
}

/* --- public functions --- */

void
tfe_window_notebook_page_new (TfeWindow *win) {
  notebook_page_new (win->nb);
}

void
tfe_window_notebook_page_new_with_files (TfeWindow *win, GFile **files, int n_files) {
  int i;

  for (i = 0; i < n_files; i++)
    notebook_page_new_with_file (win->nb, files[i]);
  if (gtk_notebook_get_n_pages (win->nb) == 0)
    notebook_page_new (win->nb);
}

/* --- TfeWindow object construction/destruction --- */ 
static void
tfe_window_dispose (GObject *gobject) {
  TfeWindow *window = TFE_WINDOW (gobject);

  g_clear_object (&window->settings);
  G_OBJECT_CLASS (tfe_window_parent_class)->dispose (gobject);
}

static void
tfe_window_init (TfeWindow *win) {
  GtkBuilder *build;
  GMenuModel *menu;

  gtk_widget_init_template (GTK_WIDGET (win));

  build = gtk_builder_new_from_resource ("/com/github/ToshioCP/tfe/menu.ui");
  menu = G_MENU_MODEL (gtk_builder_get_object (build, "menu"));
  gtk_menu_button_set_menu_model (win->btnm, menu);
  g_object_unref(build);

  win->settings = g_settings_new ("com.github.ToshioCP.tfe");
  g_signal_connect (win->settings, "changed::font", G_CALLBACK (changed_font_cb), win);

/* ----- action ----- */
  const GActionEntry win_entries[] = {
    { "open", open_activated, NULL, NULL, NULL },
    { "save", save_activated, NULL, NULL, NULL },
    { "close", close_activated, NULL, NULL, NULL },
    { "new", new_activated, NULL, NULL, NULL },
    { "saveas", saveas_activated, NULL, NULL, NULL },
    { "pref", pref_activated, NULL, NULL, NULL },
    { "close-all", quit_activated, NULL, NULL, NULL }
  };
  g_action_map_add_action_entries (G_ACTION_MAP (win), win_entries, G_N_ELEMENTS (win_entries), win);

  changed_font_cb(win->settings, "font", win);
}

static void
tfe_window_class_init (TfeWindowClass *class) {
  GObjectClass *object_class = G_OBJECT_CLASS (class);

  object_class->dispose = tfe_window_dispose;
  gtk_widget_class_set_template_from_resource (GTK_WIDGET_CLASS (class), "/com/github/ToshioCP/tfe/tfewindow.ui");
  gtk_widget_class_bind_template_child (GTK_WIDGET_CLASS (class), TfeWindow, btnm);
  gtk_widget_class_bind_template_child (GTK_WIDGET_CLASS (class), TfeWindow, nb);
}

GtkWidget *
tfe_window_new (GtkApplication *app) {
  return GTK_WIDGET (g_object_new (TFE_TYPE_WINDOW, "application", app, NULL));
}
\end{lstlisting}

\begin{itemize}
\tightlist
\item
  17-29: \passthrough{\lstinline!alert\_response\_cb!} is a call back
  function of the ``response'' signal of TfeAlert dialog. This is the
  same as before except
  \passthrough{\lstinline!gtk\_window\_destroy(GTK\_WINDOW (win))!} is
  used instead of \passthrough{\lstinline!tfe\_application\_quit!}.
\item
  31-102: Handlers of action activated signal. The
  \passthrough{\lstinline!user\_data!} is a pointer to TfeWindow
  instance.
\item
  104-115: A handler of ``changed::font'' signal of GSettings object.
\item
  111: Gets the font from GSettings data.
\item
  112: Gets a PangoFontDescription from the font. In the previous
  version, the program gets the font description from the GtkFontButton.
  The button data and GSettings data are the same. Therefore, the data
  got here is the same as the data in the GtkFontButton. In addition, we
  don't need to worry about the preference dialog is alive or not thanks
  to the GSettings.
\item
  114: Sets CSS on the display with the font description.
\item
  117-132: Public functions.
\item
  134-141: Dispose handler. The GSettings object needs to be released.
\item
  143-171: Instance initialization function.
\item
  148: Creates a composite widget instance with the template.
\item
  151-153: Insert \passthrough{\lstinline!menu!} to the menu button.
\item
  155-156: Creates a GSettings instance with the id
  ``com.github.ToshioCP.tfe''. Connects ``changed::font'' signal to the
  handler \passthrough{\lstinline!changed\_font\_cb!}. This signal emits
  when the GSettings data is changed. The second part ``font'' of the
  signal name ``changed::font'' is called details. Signals can have
  details. If a GSettings instance has more than one key, ``changed''
  signal emits only if the key, which has the same name as the detail,
  changes its value. For example, Suppose a GSettings object has three
  keys ``a'', ``b'' and ``c''.

  \begin{itemize}
  \tightlist
  \item
    ``changed::a'' is emitted when the value of the key ``a'' is
    changed. It isn't emitted when the value of ``b'' or ``c'' is
    changed.
  \item
    ``changed::b'' is emitted when the value of the key ``b'' is
    changed. It isn't emitted when the value of ``a'' or ``c'' is
    changed.
  \item
    ``changed::c'' is emitted when the value of the key ``c'' is
    changed. It isn't emitted when the value of ``a'' or ``b'' is
    changed. In this version of tfe, there is only one key (``font'').
    So, even if the signal doesn't have a detail, the result is the
    same. But in the future version, it will probably need details.
  \end{itemize}
\item
  158-168: Creates actions.
\item
  170: Sets CSS font.
\item
  173-181: Class initialization function.
\item
  177: Sets the dispose handler.
\item
  178: Sets the composite widget template
\item
  179-180: Binds private variable with child objects in the template.
\item
  183-186: \passthrough{\lstinline!tfe\_window\_new!}. This function
  creates TfeWindow instance.
\end{itemize}

\hypertarget{tfeapplication}{%
\subsection{TfeApplication}\label{tfeapplication}}

The file \passthrough{\lstinline!tfeapplication.c!} is now very simple.

\begin{lstlisting}[language=C, numbers=left]
#include "tfewindow.h"

/* ----- activate, open, startup handlers ----- */
static void
app_activate (GApplication *application) {
  GtkApplication *app = GTK_APPLICATION (application);
  GtkWidget *win = GTK_WIDGET (gtk_application_get_active_window (app));

  tfe_window_notebook_page_new (TFE_WINDOW (win));
  gtk_widget_show (GTK_WIDGET (win));
}

static void
app_open (GApplication *application, GFile ** files, gint n_files, const gchar *hint) {
  GtkApplication *app = GTK_APPLICATION (application);
  GtkWidget *win = GTK_WIDGET (gtk_application_get_active_window (app));

  tfe_window_notebook_page_new_with_files (TFE_WINDOW (win), files, n_files);
  gtk_widget_show (win);
}

static void
app_startup (GApplication *application) {
  GtkApplication *app = GTK_APPLICATION (application);
  int i;

  tfe_window_new (app);

/* ----- accelerator ----- */ 
  struct {
    const char *action;
    const char *accels[2];
  } action_accels[] = {
    { "win.open", { "<Control>o", NULL } },
    { "win.save", { "<Control>s", NULL } },
    { "win.close", { "<Control>w", NULL } },
    { "win.new", { "<Control>n", NULL } },
    { "win.saveas", { "<Shift><Control>s", NULL } },
    { "win.close-all", { "<Control>q", NULL } },
  };

  for (i = 0; i < G_N_ELEMENTS(action_accels); i++)
    gtk_application_set_accels_for_action(GTK_APPLICATION(app), action_accels[i].action, action_accels[i].accels);
}

/* ----- main ----- */

#define APPLICATION_ID "com.github.ToshioCP.tfe"

int
main (int argc, char **argv) {
  GtkApplication *app;
  int stat;

  app = gtk_application_new (APPLICATION_ID, G_APPLICATION_HANDLES_OPEN);

  g_signal_connect (app, "startup", G_CALLBACK (app_startup), NULL);
  g_signal_connect (app, "activate", G_CALLBACK (app_activate), NULL);
  g_signal_connect (app, "open", G_CALLBACK (app_open), NULL);

  stat =g_application_run (G_APPLICATION (app), argc, argv);
  g_object_unref (app);
  return stat;
}
\end{lstlisting}

\begin{itemize}
\tightlist
\item
  4-11: Activate signal handler. It uses
  \passthrough{\lstinline!tfe\_window\_notebook\_page\_new!} instead of
  \passthrough{\lstinline!notebook\_page\_new!}.
\item
  13-20: Open signal handler. Thanks to
  \passthrough{\lstinline!tfe\_window\_notebook\_page\_new\_with\_files!},
  this handler becomes very simple.
\item
  22-44: Startup signal handler. Most of the tasks are moved to
  TfeWindow, the remaining tasks are creating a window and setting
  accelerations.
\item
  48-64: A function \passthrough{\lstinline!main!}.
\end{itemize}

\hypertarget{other-files}{%
\subsection{Other files}\label{other-files}}

Resource XML file.

\begin{lstlisting}[language=XML, numbers=left]
<?xml version="1.0" encoding="UTF-8"?>
<gresources>
  <gresource prefix="/com/github/ToshioCP/tfe">
    <file>tfewindow.ui</file>
    <file>tfepref.ui</file>
    <file>tfealert.ui</file>
    <file>menu.ui</file>
  </gresource>
</gresources>
\end{lstlisting}

GSchema XML file

\begin{lstlisting}[language=XML, numbers=left]
<?xml version="1.0" encoding="UTF-8"?>
<schemalist>
  <schema path="/com/github/ToshioCP/tfe/" id="com.github.ToshioCP.tfe">
    <key name="font" type="s">
      <default>'Monospace 12'</default>
      <summary>Font</summary>
      <description>The font to be used for textview.</description>
    </key>
  </schema>
</schemalist>
\end{lstlisting}

Meson.build

\begin{lstlisting}[numbers=left]
project('tfe', 'c')

gtkdep = dependency('gtk4')

gnome=import('gnome')
resources = gnome.compile_resources('resources','tfe.gresource.xml')
gnome.compile_schemas(build_by_default: true, depend_files: 'com.github.ToshioCP.tfe.gschema.xml')

sourcefiles=files('tfeapplication.c', 'tfewindow.c', 'tfenotebook.c', 'tfepref.c', 'tfealert.c', 'css.c', '../tfetextview/tfetextview.c')

executable('tfe', sourcefiles, resources, dependencies: gtkdep, export_dynamic: true, install: true)

schema_dir = get_option('prefix') / get_option('datadir') / 'glib-2.0/schemas/'
install_data('com.github.ToshioCP.tfe.gschema.xml', install_dir: schema_dir)
meson.add_install_script('glib-compile-schemas', schema_dir)
\end{lstlisting}

\hypertarget{compilation-and-installation.}{%
\subsection{Compilation and
installation.}\label{compilation-and-installation.}}

If you build Gtk4 from the source, use
\passthrough{\lstinline!--prefix!} option.

\begin{lstlisting}
$ meson --prefix=$HOME/local _build
$ ninja -C _build
$ ninja -C _build install
\end{lstlisting}

If you install Gtk4 from the distribution packages, you don't need the
prefix option. Maybe you need root privilege to install it.

\begin{lstlisting}
$ meson _build
$ ninja -C _build
$ ninja -C _build install  # or 'sudo ninja -C _build install'
\end{lstlisting}

Source files are in src/tfe7 directory.

We made a very small text editor. You can add features to this editor.
When you add a new feature, care about the structure of the program.
Maybe you need to divide a file into several files like this section. It
isn't good to put many things into one file. And it is important to
think about the relationship between source files and widget structures.
It is appropriate that they correspond to each other in many cases.
