\hypertarget{contenido-de-este-repositorio}{%
\paragraph{Contenido de este
repositorio}\label{contenido-de-este-repositorio}}

Este tutorial ilustra como escribir programas en C con la biblioteca
Gtk4 se enfoca en prinpcipiantes, asi que el contenido esta limitado a
los temas básicos. La tabla de contenido se encuentra al final de esta
introducción.

\begin{itemize}
\tightlist
\item
  Sección 3 a la 21 describe las bases, con el ejemplo de un editor
  simple \passthrough{\lstinline!tfe!} (Text File Editor).
\item
  Sección 22 a la 25 describe como usar GtkDrawingArea.
\item
  Sección 26 a la 29 describe el modelo de lista y la vista lista
  (GtkListView, GtkGridView y GtkColumnView). también describe
  GtkExpression.
\end{itemize}

La última versión original de este tutorial (en inglés) se encuentra en
\href{https://github.com/ToshioCP/Gtk4-tutorial}{Gtk4-tutorial github
repository}. Puedes leerlo desde ahí directamente sin tener que
descargar nada.

\hypertarget{documentaciuxf3n-gtk4}{%
\paragraph{Documentación Gtk4}\label{documentaciuxf3n-gtk4}}

Lee \href{https://docs.gtk.org/gtk4/index.html}{Gtk API Reference} y and
\href{https://developer.gnome.org/}{Gnome Developer Documentation
Website} para mayor información.

Estos sitios son recientes (Agosto 2021) La vieja documentación se
encuentra en \href{https://developer-old.gnome.org/gtk4/stable/}{Gtk
Reference Manual} y \href{https://developer-old.gnome.org/}{Gnome
Developer Center}. El nuevo sitio se encuentra en progreso actualmente,
asi que a veces tendrás que revisar la vieja version

Si deseas conocer mas acerca de GObject y el sistema de tipos, por favor
lee \href{https://github.com/ToshioCP/Gobject-tutorial}{GObject
tutorial}. Los detalles de GObject son fáciles de entender y necesarios
para escribir programas en Gtk4.

\hypertarget{contribuciuxf3n}{%
\paragraph{Contribución}\label{contribuciuxf3n}}

Este tutorial se encuentra bajo desarrollo y es inestable. Incluso
aunque los ejemplos han sido probados en Gtk4 versión 4.0, pueden
existir algunos errores. Si encuentras algún bug, o errores en el
tutorial y los ejemplos de C, por favor haznoslo notar. Lo puedes
publicar en (sitio original en inglés)
\href{https://github.com/ToshioCP/Gtk4-tutorial/issues}{github issues}.
También puedes publicar los archivos corregidos como un commit a
\href{https://github.com/ToshioCP/Gtk4-tutorial/pulls}{pull request}.
Cuando hagas correcciones, corrige el archivo fuente, que se encuentra
en la carpeta `src', y ejecuta \passthrough{\lstinline!rake!}para crear
el archivo de salida. Los archivos GFM dentro de la carpeta `gfm' se
actualizan de manera automática.

Si tienes alguna duda, puedes publicarlo como un issue dentro del sitio.
Todas las preguntas son útiles y harán este tutorial mejor.

\hypertarget{como-obtener-una-versiuxf3n-html-o-pdf}{%
\paragraph{Como obtener una versión HTML o
PDF}\label{como-obtener-una-versiuxf3n-html-o-pdf}}

Si deseas obtener una versión HTML o PDF, la puedes hacer com
\passthrough{\lstinline!rake!}, que es una versión en ruby de make.
Escribe \passthrough{\lstinline!rake html!} para HTML. Escribe
\passthrough{\lstinline!rake pdf!} para PDF. El apéndice ``Cómo
construir el Tutorial Gtk4'' describe como hacerlo.
