\hypertarget{tfeapplication.c}{%
\section{tfeapplication.c}\label{tfeapplication.c}}

\passthrough{\lstinline!tfeapplication.c!} includes all the code other
than \passthrough{\lstinline!tfetxtview.c!} and
\passthrough{\lstinline!tfenotebook.c!}. It does:

\begin{itemize}
\tightlist
\item
  Application support, mainly handling command line arguments.
\item
  Builds widgets using ui file.
\item
  Connects button signals and their handlers.
\item
  Manages CSS.
\end{itemize}

\hypertarget{main}{%
\subsection{main}\label{main}}

The function \passthrough{\lstinline!main!} is the first invoked
function in C language. It connects the command line given by the user
and Gtk application.

\begin{lstlisting}[language=C, numbers=left]
#define APPLICATION_ID "com.github.ToshioCP.tfe"

int
main (int argc, char **argv) {
  GtkApplication *app;
  int stat;

  app = gtk_application_new (APPLICATION_ID, G_APPLICATION_HANDLES_OPEN);

  g_signal_connect (app, "startup", G_CALLBACK (app_startup), NULL);
  g_signal_connect (app, "activate", G_CALLBACK (app_activate), NULL);
  g_signal_connect (app, "open", G_CALLBACK (app_open), NULL);

  stat =g_application_run (G_APPLICATION (app), argc, argv);
  g_object_unref (app);
  return stat;
}
\end{lstlisting}

\begin{itemize}
\tightlist
\item
  1: Defines the application id. It is easy to find the application id,
  and better than the id is embedded in
  \passthrough{\lstinline!gtk\_application\_new!}.
\item
  8: Creates GtkApplication object.
\item
  10-12: Connects ``startup'', ``activate'' and ``open'' signals to
  their handlers.
\item
  14: Runs the application.
\item
  15-16: releases the reference to the application and returns the
  status.
\end{itemize}

\hypertarget{startup-signal-handler}{%
\subsection{startup signal handler}\label{startup-signal-handler}}

Startup signal is emitted just after the GtkApplication instance is
initialized. What the signal handler needs to do is initialization of
the application.

\begin{itemize}
\tightlist
\item
  Builds the widgets using ui file.
\item
  Connects button signals and their handlers.
\item
  Sets CSS.
\end{itemize}

The handler is as follows.

\begin{lstlisting}[language=C, numbers=left]
static void
app_startup (GApplication *application) {
  GtkApplication *app = GTK_APPLICATION (application);
  GtkBuilder *build;
  GtkApplicationWindow *win;
  GtkNotebook *nb;
  GtkButton *btno;
  GtkButton *btnn;
  GtkButton *btns;
  GtkButton *btnc;

  build = gtk_builder_new_from_resource ("/com/github/ToshioCP/tfe/tfe.ui");
  win = GTK_APPLICATION_WINDOW (gtk_builder_get_object (build, "win"));
  nb = GTK_NOTEBOOK (gtk_builder_get_object (build, "nb"));
  gtk_window_set_application (GTK_WINDOW (win), app);
  btno = GTK_BUTTON (gtk_builder_get_object (build, "btno"));
  btnn = GTK_BUTTON (gtk_builder_get_object (build, "btnn"));
  btns = GTK_BUTTON (gtk_builder_get_object (build, "btns"));
  btnc = GTK_BUTTON (gtk_builder_get_object (build, "btnc"));
  g_signal_connect_swapped (btno, "clicked", G_CALLBACK (open_cb), nb);
  g_signal_connect_swapped (btnn, "clicked", G_CALLBACK (new_cb), nb);
  g_signal_connect_swapped (btns, "clicked", G_CALLBACK (save_cb), nb);
  g_signal_connect_swapped (btnc, "clicked", G_CALLBACK (close_cb), nb);
  g_object_unref(build);

GdkDisplay *display;

  display = gtk_widget_get_display (GTK_WIDGET (win));
  GtkCssProvider *provider = gtk_css_provider_new ();
  gtk_css_provider_load_from_data (provider, "textview {padding: 10px; font-family: monospace; font-size: 12pt;}", -1);
  gtk_style_context_add_provider_for_display (display, GTK_STYLE_PROVIDER (provider), GTK_STYLE_PROVIDER_PRIORITY_APPLICATION);
}
\end{lstlisting}

\begin{itemize}
\tightlist
\item
  12-15: Builds widgets using ui file (resource). Connects the top-level
  window and the application with
  \passthrough{\lstinline!gtk\_window\_set\_application!}.
\item
  16-23: Gets buttons and connects their signals and handlers.
\item
  24: Releases the reference to GtkBuilder.
\item
  26-31: Sets CSS. CSS in Gtk is similar to CSS in HTML. You can set
  margin, border, padding, color, font and so on with CSS. In this
  program CSS is in line 30. It sets padding, font-family and font size
  of GtkTextView.
\item
  26-28: GdkDisplay is used to set CSS. CSS will be explained in the
  next subsection.
\end{itemize}

\hypertarget{css-in-gtk}{%
\subsection{CSS in Gtk}\label{css-in-gtk}}

CSS is an abbreviation of Cascading Style Sheet. It is originally used
with HTML to describe the presentation semantics of a document. You
might have found that the widgets in Gtk is similar to a window in a
browser. It implies that CSS can also be applied to Gtk windowing
system.

\hypertarget{css-nodes-selectors}{%
\subsubsection{CSS nodes, selectors}\label{css-nodes-selectors}}

The syntax of CSS is as follows.

\begin{lstlisting}
selector { color: yellow; padding-top: 10px; ...}
\end{lstlisting}

Every widget has CSS node. For example GtkTextView has
\passthrough{\lstinline!textview!} node. If you want to set style to
GtkTextView, substitute ``textview'' for the selector.

\begin{lstlisting}
textview {color: yellow; ...}
\end{lstlisting}

Class, ID and some other things can be applied to the selector like Web
CSS. Refer to \href{https://docs.gtk.org/gtk4/css-overview.html}{Gtk4
API Reference, CSS in Gtk} for further information.

In line 30, the CSS is a string.

\begin{lstlisting}
textview {padding: 10px; font-family: monospace; font-size: 12pt;}
\end{lstlisting}

\begin{itemize}
\tightlist
\item
  padding is a space between the border and contents. This space makes
  the textview easier to read.
\item
  font-family is a name of font. ``monospace'' is one of the generic
  family font keywords.
\item
  font-size is set to 12pt.
\end{itemize}

\hypertarget{gtkstylecontext-gtkcssprovider-and-gdkdisplay}{%
\subsubsection{GtkStyleContext, GtkCSSProvider and
GdkDisplay}\label{gtkstylecontext-gtkcssprovider-and-gdkdisplay}}

GtkStyleContext is an object that stores styling information affecting a
widget. Each widget is connected to the corresponding GtkStyleContext.
You can get the context by
\passthrough{\lstinline!gtk\_widget\_get\_style\_context!}.

GtkCssProvider is an object which parses CSS in order to style widgets.

To apply your CSS to widgets, you need to add GtkStyleProvider (the
interface of GtkCSSProvider) to GtkStyleContext. However, instead, you
can add it to GdkDisplay of the window (usually top-level window).

Look at the source file of \passthrough{\lstinline!startup!} handler
again.

\begin{itemize}
\tightlist
\item
  28: The display is obtained by
  \passthrough{\lstinline!gtk\_widget\_get\_display!}.
\item
  29: Creates a GtkCssProvider instance.
\item
  30: Puts the CSS into the provider.
\item
  31: Adds the provider to the display. The last argument of
  \passthrough{\lstinline!gtk\_style\_context\_add\_provider\_for\_display!}
  is the priority of the style provider.
  \passthrough{\lstinline!GTK\_STYLE\_PROVIDER\_PRIORITY\_APPLICATION!}
  is a priority for application-specific style information.
  \passthrough{\lstinline!GTK\_STYLE\_PROVIDER\_PRIORITY\_USER!} is also
  often used and it is the highest priority. So,
  \passthrough{\lstinline!GTK\_STYLE\_PROVIDER\_PRIORITY\_USER!} is
  often used to a specific widget.
\end{itemize}

It is possible to add the provider to the context of GtkTextView instead
of GdkDiplay. To do so, rewrite
\passthrough{\lstinline!tfe\_text\_view\_new!}. First, get the
GtkStyleContext object of a TfeTextView object. Then adds the CSS
provider to the context.

\begin{lstlisting}[language=C]
GtkWidget *
tfe_text_view_new (void) {
  GtkWidget *tv;

  tv = gtk_widget_new (TFE_TYPE_TEXT_VIEW, NULL);

  GtkStyleContext *context;

  context = gtk_widget_get_style_context (GTK_WIDGET (tv));
  GtkCssProvider *provider = gtk_css_provider_new ();
  gtk_css_provider_load_from_data (provider, "textview {padding: 10px; font-family: monospace; font-size: 12pt;}", -1);
  gtk_style_context_add_provider (context, GTK_STYLE_PROVIDER (provider), GTK_STYLE_PROVIDER_PRIORITY_USER);

  return tv;
}
\end{lstlisting}

CSS in the context takes precedence over CSS in the display.

\hypertarget{activate-and-open-handler}{%
\subsection{activate and open handler}\label{activate-and-open-handler}}

The handler of ``activate'' and ``open'' signal are
\passthrough{\lstinline!app\_activate!} and
\passthrough{\lstinline!app\_open!} respectively. They just create a new
GtkNotebookPage.

\begin{lstlisting}[language=C, numbers=left]
static void
app_activate (GApplication *application) {
  GtkApplication *app = GTK_APPLICATION (application);
  GtkWidget *win = GTK_WIDGET (gtk_application_get_active_window (app));
  GtkWidget *boxv = gtk_window_get_child (GTK_WINDOW (win));
  GtkNotebook *nb = GTK_NOTEBOOK (gtk_widget_get_last_child (boxv));

  notebook_page_new (nb);
  gtk_widget_show (GTK_WIDGET (win));
}

static void
app_open (GApplication *application, GFile ** files, gint n_files, const gchar *hint) {
  GtkApplication *app = GTK_APPLICATION (application);
  GtkWidget *win = GTK_WIDGET (gtk_application_get_active_window (app));
  GtkWidget *boxv = gtk_window_get_child (GTK_WINDOW (win));
  GtkNotebook *nb = GTK_NOTEBOOK (gtk_widget_get_last_child (boxv));
  int i;

  for (i = 0; i < n_files; i++)
    notebook_page_new_with_file (nb, files[i]);
  if (gtk_notebook_get_n_pages (nb) == 0)
    notebook_page_new (nb);
  gtk_widget_show (win);
}
\end{lstlisting}

\begin{itemize}
\tightlist
\item
  1-10: \passthrough{\lstinline!app\_activate!}.
\item
  8-10: Creates a new page and shows the window.
\item
  12-25: \passthrough{\lstinline!app\_open!}.
\item
  20-21: Creates notebook pages with files.
\item
  22-23: If no page has created, maybe because of read error, then it
  creates an empty page.
\item
  24: Shows the window.
\end{itemize}

These codes have become really simple thanks to tfenotebook.c and
tfetextview.c.

\hypertarget{primary-instance}{%
\subsection{Primary instance}\label{primary-instance}}

Only one GApplication instance can be run at a time per session. The
session is a bit difficult concept and also platform-dependent, but
roughly speaking, it corresponds to a graphical desktop login. When you
use your PC, you probably login first, then your desktop appears until
you log off. This is the session.

However, Linux is multi process OS and you can run two or more instances
of the same application. Isn't it a contradiction?

When first instance is launched, then it registers itself with its
application ID (for example,
\passthrough{\lstinline!com.github.ToshioCP.tfe!}). Just after the
registration, startup signal is emitted, then activate or open signal is
emitted and the instance's main loop runs. I wrote ``startup signal is
emitted just after the application instance is initialized'' in the
prior subsection. More precisely, it is emitted just after the
registration.

If another instance which has the same application ID is invoked, it
also tries to register itself. Because this is the second instance, the
registration of the ID has already done, so it fails. Because of the
failure startup signal isn't emitted. After that, activate or open
signal is emitted in the primary instance, not the second instance. The
primary instance receives the signal and its handler is invoked. On the
other hand, the second instance doesn't receive the signal and it
immediately quits.

Try to run two instances in a row.

\begin{lstlisting}
$ ./_build/tfe &
[1] 84453
$ ./build/tfe tfeapplication.c
$
\end{lstlisting}

First, the primary instance opens a window. Then, after the second
instance is run, a new notebook page with the contents of
\passthrough{\lstinline!tfeapplication.c!} appears in the primary
instance's window. This is because the open signal is emitted in the
primary instance. The second instance immediately quits so shell prompt
soon appears.

\hypertarget{a-series-of-handlers-correspond-to-the-button-signals}{%
\subsection{a series of handlers correspond to the button
signals}\label{a-series-of-handlers-correspond-to-the-button-signals}}

\begin{lstlisting}[language=C, numbers=left]
static void
open_cb (GtkNotebook *nb) {
  notebook_page_open (nb);
}

static void
new_cb (GtkNotebook *nb) {
  notebook_page_new (nb);
}

static void
save_cb (GtkNotebook *nb) {
  notebook_page_save (nb);
}

static void
close_cb (GtkNotebook *nb) {
  notebook_page_close (GTK_NOTEBOOK (nb));
}
\end{lstlisting}

\passthrough{\lstinline!open\_cb!}, \passthrough{\lstinline!new\_cb!},
\passthrough{\lstinline!save\_cb!} and
\passthrough{\lstinline!close\_cb!} just call corresponding notebook
page functions.

\hypertarget{meson.build}{%
\subsection{meson.build}\label{meson.build}}

\begin{lstlisting}[numbers=left]
project('tfe', 'c')

gtkdep = dependency('gtk4')

gnome=import('gnome')
resources = gnome.compile_resources('resources','tfe.gresource.xml')

sourcefiles=files('tfeapplication.c', 'tfenotebook.c', '../tfetextview/tfetextview.c')

executable('tfe', sourcefiles, resources, dependencies: gtkdep)
\end{lstlisting}

In this file, just the source file names are modified from the prior
version.

\hypertarget{source-files}{%
\subsection{source files}\label{source-files}}

The source files of the text editor \passthrough{\lstinline!tfe!} will
be shown in the next section.

You can also download the files from the
\href{https://github.com/ToshioCP/Gtk4-tutorial}{repository}. There are
two options.

\begin{itemize}
\tightlist
\item
  Use git and clone.
\item
  Run your browser and open the
  \href{https://github.com/ToshioCP/Gtk4-tutorial}{top page}. Then click
  on ``Code'' button and click ``Download ZIP'' in the popup menu. After
  that, unzip the archive file.
\end{itemize}

If you use git, run the terminal and type the following.

\begin{lstlisting}
$ git clone https://github.com/ToshioCP/Gtk4-tutorial.git
\end{lstlisting}

The source files are under /src/tfe5 directory.
