\hypertarget{functions-in-gtknotebook}{%
\section{Functions in GtkNotebook}\label{functions-in-gtknotebook}}

GtkNotebook is a very important object in the text file editor
\passthrough{\lstinline!tfe!}. It connects the application and
TfeTextView objects. A set of public functions are declared in
\passthrough{\lstinline!tfenotebook.h!}. The word ``tfenotebook'' is
used only in filenames. There's no ``TfeNotebook'' object.

\begin{lstlisting}[language=C, numbers=left]
void
notebook_page_save(GtkNotebook *nb);

void
notebook_page_close (GtkNotebook *nb);

void
notebook_page_open (GtkNotebook *nb);

void
notebook_page_new_with_file (GtkNotebook *nb, GFile *file);

void
notebook_page_new (GtkNotebook *nb);
\end{lstlisting}

This header file describes the public functions in
\passthrough{\lstinline!tfenotebook.c!}.

\begin{itemize}
\tightlist
\item
  1-2: \passthrough{\lstinline!notebook\_page\_save!} saves the current
  page to the file of which the name specified in the tab. If the name
  is \passthrough{\lstinline!untitled!} or
  \passthrough{\lstinline!untitled!} followed by digits,
  FileChooserDialog appears and a user can choose or specify a filename.
\item
  4-5: \passthrough{\lstinline!notebook\_page\_close!} closes the
  current page.
\item
  7-8: \passthrough{\lstinline!notebook\_page\_open!} shows a file
  chooser dialog and a user can choose a file. The file is inserted to a
  new page.
\item
  10-11: \passthrough{\lstinline!notebook\_page\_new\_with\_file!}
  creates a new page and the file given as an argument is read and
  inserted into the page.
\item
  13-14: \passthrough{\lstinline!notebook\_page\_new!} creates a new
  empty page.
\end{itemize}

You probably find that the functions except
\passthrough{\lstinline!notebook\_page\_close!} are higher level
functions of

\begin{itemize}
\tightlist
\item
  \passthrough{\lstinline!tfe\_text\_view\_save!}
\item
  \passthrough{\lstinline!tef\_text\_view\_open!}
\item
  \passthrough{\lstinline!tfe\_text\_view\_new\_with\_file!}
\item
  \passthrough{\lstinline!tfe\_text\_view\_new!}
\end{itemize}

respectively.

There are two layers. One of them is
\passthrough{\lstinline!tfe\_text\_view ...!}, which is the lower level
layer. The other is \passthrough{\lstinline!note\_book ...!}, which is
the higher level layer.

Now let's look at the program of each function.

\hypertarget{notebook_page_new}{%
\subsection{notebook\_page\_new}\label{notebook_page_new}}

\begin{lstlisting}[language=C, numbers=left]
static gchar*
get_untitled () {
  static int c = -1;
  if (++c == 0) 
    return g_strdup_printf("Untitled");
  else
    return g_strdup_printf ("Untitled%u", c);
}

static void
notebook_page_build (GtkNotebook *nb, GtkWidget *tv, char *filename) {
  GtkWidget *scr = gtk_scrolled_window_new ();
  GtkNotebookPage *nbp;
  GtkWidget *lab;
  int i;

  gtk_text_view_set_wrap_mode (GTK_TEXT_VIEW (tv), GTK_WRAP_WORD_CHAR);
  gtk_scrolled_window_set_child (GTK_SCROLLED_WINDOW (scr), tv);
  lab = gtk_label_new (filename);
  i = gtk_notebook_append_page (nb, scr, lab);
  nbp = gtk_notebook_get_page (nb, scr);
  g_object_set (nbp, "tab-expand", TRUE, NULL);
  gtk_notebook_set_current_page (nb, i);
  g_signal_connect (GTK_TEXT_VIEW (tv), "change-file", G_CALLBACK (file_changed_cb), nb);
}

void
notebook_page_new (GtkNotebook *nb) {
  g_return_if_fail(GTK_IS_NOTEBOOK (nb));

  GtkWidget *tv;
  char *filename;

  if ((tv = tfe_text_view_new ()) == NULL)
    return;
  filename = get_untitled ();
  notebook_page_build (nb, tv, filename);
}
\end{lstlisting}

\begin{itemize}
\tightlist
\item
  27-38: \passthrough{\lstinline!notebook\_page\_new!} function.
\item
  29: \passthrough{\lstinline!g\_return\_if\_fail!} is used to check the
  argument.
\item
  34: Creates TfeTextView object. If it fails, it returns to the caller.
\item
  36: Creates filename, which is ``Untitled'', ``Untitled1'', \ldots{} .
\item
  1-8: \passthrough{\lstinline!get\_untitled!} function.
\item
  3: Static variable \passthrough{\lstinline!c!} is initialized at the
  first call of this function. After that \passthrough{\lstinline!c!}
  keeps its value unless it is changed explicitly.
\item
  4-7: Increases \passthrough{\lstinline!c!} by one and if it is zero
  then it returns ``Untitled''. If it is a positive integer then it
  returns ``Untitled\textless the integer\textgreater{}'', for example,
  ``Untitled1'', ``Untitled2'', and so on. The function
  \passthrough{\lstinline!g\_strdup\_printf!} creates a string and it
  should be freed by \passthrough{\lstinline!g\_free!} when it becomes
  useless. The caller of \passthrough{\lstinline!get\_untitled!} is in
  charge of freeing the string.
\item
  37: calls \passthrough{\lstinline!notebook\_page\_build!} to build the
  contents of the page.
\item
  10- 25: \passthrough{\lstinline!notebook\_page\_build!} function.
\item
  12: Creates GtkScrolledWindow.
\item
  17: Sets the wrap mode of \passthrough{\lstinline!tv!} to
  GTK\_WRAP\_WORD\_CHAR so that lines are broken between words or
  graphemes.
\item
  18: Inserts \passthrough{\lstinline!tv!} to GtkscrolledWindow as a
  child.
\item
  19-20: Creates GtkLabel, then appends \passthrough{\lstinline!scr!}
  and \passthrough{\lstinline!lab!} to the GtkNotebook instance
  \passthrough{\lstinline!nb!}.
\item
  21-22: Sets ``tab-expand'' property to TRUE. The function
  \passthrough{\lstinline!g\_object\_set!} sets properties on an object.
  The object is any object derived from GObject. In many cases, an
  object has its own function to set its properties, but sometimes not.
  In that case, use \passthrough{\lstinline!g\_object\_set!} to set the
  property.
\item
  23: Sets the current page of \passthrough{\lstinline!nb!} to the newly
  created page.
\item
  24: Connects ``change-file'' signal and
  \passthrough{\lstinline!file\_changed\_cb!} handler.
\end{itemize}

\hypertarget{notebook_page_new_with_file}{%
\subsection{notebook\_page\_new\_with\_file}\label{notebook_page_new_with_file}}

\begin{lstlisting}[language=C, numbers=left]
void
notebook_page_new_with_file (GtkNotebook *nb, GFile *file) {
  g_return_if_fail(GTK_IS_NOTEBOOK (nb));
  g_return_if_fail(G_IS_FILE (file));

  GtkWidget *tv;
  char *filename;

  if ((tv = tfe_text_view_new_with_file (file)) == NULL)
    return; /* read error */
  filename = g_file_get_basename (file);
  notebook_page_build (nb, tv, filename);
}
\end{lstlisting}

\begin{itemize}
\tightlist
\item
  9-10: Calls
  \passthrough{\lstinline!tfe\_text\_view\_new\_with\_file!}. If the
  function returns NULL, an error has happend. Then, it does nothing and
  returns.
\item
  11-12: Gets the filename and builds the contents of the page.
\end{itemize}

\hypertarget{notebook_page_open}{%
\subsection{notebook\_page\_open}\label{notebook_page_open}}

\begin{lstlisting}[language=C, numbers=left]
static void
open_response (TfeTextView *tv, int response, GtkNotebook *nb) {
  GFile *file;
  char *filename;

  if (response != TFE_OPEN_RESPONSE_SUCCESS || ! G_IS_FILE (file = tfe_text_view_get_file (tv))) {
    g_object_ref_sink (tv);
    g_object_unref (tv);
  }else {
    filename = g_file_get_basename (file);
    g_object_unref (file);
    notebook_page_build (nb, GTK_WIDGET (tv), filename);
  }
}

void
notebook_page_open (GtkNotebook *nb) {
  g_return_if_fail(GTK_IS_NOTEBOOK (nb));

  GtkWidget *tv;

  if ((tv = tfe_text_view_new ()) == NULL)
    return;
  g_signal_connect (TFE_TEXT_VIEW (tv), "open-response", G_CALLBACK (open_response), nb);
  tfe_text_view_open (TFE_TEXT_VIEW (tv), GTK_WINDOW (gtk_widget_get_ancestor (GTK_WIDGET (nb), GTK_TYPE_WINDOW)));
}
\end{lstlisting}

\begin{itemize}
\tightlist
\item
  16-26: \passthrough{\lstinline!notebook\_page\_open!} function.
\item
  22-23: Creates TfeTextView object. If NULL is returned, an error has
  happened. Then, it returns to the caller.
\item
  24: Connects the signal ``open-response'' and the handler
  \passthrough{\lstinline!open\_response!}.
\item
  25: Calls \passthrough{\lstinline!tfe\_text\_view\_open!}. The
  ``open-response'' signal will be emitted later to inform the result of
  opening and reading a file.
\item
  1-14: \passthrough{\lstinline!open\_response!} handler.
\item
  6-8: If the response code is NOT
  \passthrough{\lstinline!TFE\_OPEN\_RESPONSE\_SUCCESS!} or
  \passthrough{\lstinline!tfe\_text\_view\_get\_file!} doesn't return
  the pointer to a GFile, it has failed to open and read a new file.
  Then, what \passthrough{\lstinline!notebook\_page\_open!} did in
  advance need to be canceled. The instance \passthrough{\lstinline!tv!}
  hasn't been a child widget of GtkScrolledWindow yet. Such instance has
  floating reference. Floating reference will be explained later in this
  subsection. You need to call
  \passthrough{\lstinline!g\_object\_ref\_sink!} first. Then the
  floating reference is converted into an ordinary reference. Now you
  call \passthrough{\lstinline!g\_object\_unref!} to decrease the
  reference count by one.
\item
  9-13: Otherwise, everything is okay. Gets the filename, builds the
  contents of the page.
\end{itemize}

All the widgets are derived from GInitiallyUnowned. When an instance of
GInitiallyUnowned or its descendant is created, the instance has a
floating reference. The function
\passthrough{\lstinline!g\_object\_ref\_sink!} converts the floating
reference into an ordinary reference. If the instance doesn't have a
floating reference, \passthrough{\lstinline!g\_object\_ref\_sink!}
simply increases the reference count by one. On the other hand, when an
instance of GObject (not GInitiallyUnowned) is created, no floating
reference is given. And the instance has a normal reference count
instead of floating reference.

If you use \passthrough{\lstinline!g\_object\_unref!} to an instance
that has a floating reference, you need to convert the floating
reference to a normal reference in advance. See
\href{https://developer-old.gnome.org/gobject/stable/gobject-The-Base-Object-Type.html\#gobject-The-Base-Object-Type.description}{GObject
Reference Manual} for further information.

\hypertarget{notebook_page_close}{%
\subsection{notebook\_page\_close}\label{notebook_page_close}}

\begin{lstlisting}[language=C, numbers=left]
void
notebook_page_close (GtkNotebook *nb) {
  g_return_if_fail(GTK_IS_NOTEBOOK (nb));

  GtkWidget *win;
  int i;

  if (gtk_notebook_get_n_pages (nb) == 1) {
    win = gtk_widget_get_ancestor (GTK_WIDGET (nb), GTK_TYPE_WINDOW);
    gtk_window_destroy(GTK_WINDOW (win));
  } else {
    i = gtk_notebook_get_current_page (nb);
    gtk_notebook_remove_page (GTK_NOTEBOOK (nb), i);
  }
}
\end{lstlisting}

This function closes the current page. If the page is the only page the
notebook has, then the function destroys the top-level window and quits
the application.

\begin{itemize}
\tightlist
\item
  8-10: If the page is the only page the notebook has, it calls
  \passthrough{\lstinline!gtk\_window\_destroy!} to destroys the
  top-level window.
\item
  11-13: Otherwise, removes the current page.
\end{itemize}

\hypertarget{notebook_page_save}{%
\subsection{notebook\_page\_save}\label{notebook_page_save}}

\begin{lstlisting}[language=C, numbers=left]
static TfeTextView *
get_current_textview (GtkNotebook *nb) {
  int i;
  GtkWidget *scr;
  GtkWidget *tv;

  i = gtk_notebook_get_current_page (nb);
  scr = gtk_notebook_get_nth_page (nb, i);
  tv = gtk_scrolled_window_get_child (GTK_SCROLLED_WINDOW (scr));
  return TFE_TEXT_VIEW (tv);
}

void
notebook_page_save (GtkNotebook *nb) {
  g_return_if_fail(GTK_IS_NOTEBOOK (nb));

  TfeTextView *tv;

  tv = get_current_textview (nb);
  tfe_text_view_save (TFE_TEXT_VIEW (tv));
}
\end{lstlisting}

\begin{itemize}
\tightlist
\item
  13-21: \passthrough{\lstinline!notebook\_page\_save!}.
\item
  19: Gets TfeTextView belongs to the current page.
\item
  20: Calls \passthrough{\lstinline!tfe\_text\_view\_save!}.
\item
  1-11: \passthrough{\lstinline!get\_current\_textview!}. This function
  gets the TfeTextView object belongs to the current page.
\item
  7: Gets the page number of the current page.
\item
  8: Gets the child widget \passthrough{\lstinline!scr!}, which is a
  GtkScrolledWindow instance, of the current page.
\item
  9-10: Gets the child widget of \passthrough{\lstinline!scr!}, which is
  a TfeTextView instance, and returns it.
\end{itemize}

\hypertarget{file_changed_cb-handler}{%
\subsection{file\_changed\_cb handler}\label{file_changed_cb-handler}}

The function \passthrough{\lstinline!file\_changed\_cb!} is a handler
connected to ``change-file'' signal. If a file in a TfeTextView instance
is changed, it emits this signal. This handler changes the label of
GtkNotebookPage.

\begin{lstlisting}[language=C, numbers=left]
static void
file_changed_cb (TfeTextView *tv, GtkNotebook *nb) {
  GtkWidget *scr;
  GtkWidget *label;
  GFile *file;
  char *filename;

  file = tfe_text_view_get_file (tv);
  scr = gtk_widget_get_parent (GTK_WIDGET (tv));
  if (G_IS_FILE (file)) {
    filename = g_file_get_basename (file);
    g_object_unref (file);
  } else
    filename = get_untitled ();
  label = gtk_label_new (filename);
  g_free (filename);
  gtk_notebook_set_tab_label (nb, scr, label);
}
\end{lstlisting}

\begin{itemize}
\tightlist
\item
  8: Gets the GFile instance from \passthrough{\lstinline!tv!}.
\item
  9: Gets the GkScrolledWindow instance which is the parent widget of
  \passthrough{\lstinline!tv!}.
\item
  10-12: If \passthrough{\lstinline!file!} points GFile, then assigns
  the filename of the GFile into \passthrough{\lstinline!filename!}.
  Then, unref the GFile object \passthrough{\lstinline!file!}.
\item
  13-14: Otherwise (file is NULL), assigns untitled string to
  \passthrough{\lstinline!filename!}.
\item
  15-16: Creates a GtkLabel instance \passthrough{\lstinline!label!}
  with the filename and set the label of the GtkNotebookPage with
  \passthrough{\lstinline!label!}.
\end{itemize}
