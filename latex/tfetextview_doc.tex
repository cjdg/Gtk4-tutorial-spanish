\hypertarget{tfetextview-api-reference}{%
\section{TfeTextView API reference}\label{tfetextview-api-reference}}

TfeTextView -- Child object of GtkTextView. It holds GFile which the
contents of GtkTextBuffer correponds to.

\hypertarget{functions}{%
\subsection{Functions}\label{functions}}

\begin{itemize}
\tightlist
\item
  GFile *tfe\_text\_view\_get\_file ()
\item
  void tfe\_text\_view\_open ()
\item
  void tfe\_text\_view\_save ()
\item
  void tfe\_text\_view\_saveas ()
\item
  GtkWidget *tfe\_text\_view\_new\_with\_file ()
\item
  GtkWidget *tfe\_text\_view\_new ()
\end{itemize}

\hypertarget{signals}{%
\subsection{Signals}\label{signals}}

\begin{itemize}
\tightlist
\item
  void change-file
\item
  void open-response
\end{itemize}

\hypertarget{types-and-values}{%
\subsection{Types and Values}\label{types-and-values}}

\begin{itemize}
\tightlist
\item
  TfeTextView
\item
  TfeTextViewClass
\item
  TfeTextViewOpenResponseType
\end{itemize}

\hypertarget{object-hierarchy}{%
\subsection{Object Hierarchy}\label{object-hierarchy}}

\begin{lstlisting}
GObject
+--GInitiallyUnowned
   +--GtkWidget
      +--GtkTextView
         +--TfeTextView
\end{lstlisting}

\hypertarget{includes}{%
\subsection{Includes}\label{includes}}

\begin{lstlisting}
#include <gtk/gtk.h>
\end{lstlisting}

\hypertarget{description}{%
\subsection{Description}\label{description}}

TfeTextView holds GFile which the contents of GtkTextBuffer corresponds
to. File manipulation functions are added to this object.

\hypertarget{functions-1}{%
\subsection{Functions}\label{functions-1}}

\hypertarget{tfe_text_view_get_file}{%
\subsubsection{tfe\_text\_view\_get\_file()}\label{tfe_text_view_get_file}}

\begin{lstlisting}
GFile *
tfe_text_view_get_file (TfeTextView *tv);
\end{lstlisting}

Returns the copy of the GFile in the TfeTextView.

Parameters

\begin{itemize}
\tightlist
\item
  tv: a TfeTextView
\end{itemize}

\hypertarget{tfe_text_view_open}{%
\subsubsection{tfe\_text\_view\_open()}\label{tfe_text_view_open}}

\begin{lstlisting}
void
tfe_text_view_open (TfeTextView *tv, GtkWidget *win);
\end{lstlisting}

Just shows a GtkFileChooserDialog so that a user can choose a file to
read. This function doesn't do any I/O operations. They are done by the
signal handler connected to the \passthrough{\lstinline!response!}
signal emitted by GtkFileChooserDialog. Therefore the caller can't know
the I/O status directly from the function. Instead, the status is
informed by \passthrough{\lstinline!open-response!} signal. The caller
needs to set a handler to this signal in advance.

parameters

\begin{itemize}
\tightlist
\item
  tv: a TfeTextView
\item
  win: the top level window
\end{itemize}

\hypertarget{tfe_text_view_save}{%
\subsubsection{tfe\_text\_view\_save()}\label{tfe_text_view_save}}

\begin{lstlisting}
void
tfe_text_view_save (TfeTextView *tv);
\end{lstlisting}

Saves the contents of a TfeTextView to a file. If
\passthrough{\lstinline!tv!} holds a GFile, it is used. Otherwise, this
function shows GtkFileChosserDialog so that a user can choose a file to
save.

Parameters

\begin{itemize}
\tightlist
\item
  tv: a TfeTextView
\end{itemize}

\hypertarget{tfe_text_view_saveas}{%
\subsubsection{tfe\_text\_view\_saveas()}\label{tfe_text_view_saveas}}

\begin{lstlisting}
void
tfe_text_view_saveas (TfeTextView *tv);
\end{lstlisting}

Saves the content of a TfeTextView to a file. This function shows
GtkFileChosserDialog so that a user can choose a file to save.

Parameters

\begin{itemize}
\tightlist
\item
  tv: a TfeTextView
\end{itemize}

\hypertarget{tfe_text_view_new_with_file}{%
\subsubsection{tfe\_text\_view\_new\_with\_file()}\label{tfe_text_view_new_with_file}}

\begin{lstlisting}
GtkWidget *
tfe_text_view_new_with_file (GFile *file);
\end{lstlisting}

Creates a new TfeTextView and reads the contents of the
\passthrough{\lstinline!file!} and set it to the GtkTextBuffer
corresponds to the newly created TfeTextView. Then returns the
TfeTextView as GtkWidget. If an error happens, it returns
\passthrough{\lstinline!NULL!}.

Parameters

\begin{itemize}
\tightlist
\item
  file: a GFile
\end{itemize}

Returns

\begin{itemize}
\tightlist
\item
  a new TfeTextView.
\end{itemize}

\hypertarget{tfe_text_view_new}{%
\subsubsection{tfe\_text\_view\_new()}\label{tfe_text_view_new}}

\begin{lstlisting}
GtkWidget *
tfe_text_view_new (void);
\end{lstlisting}

Creates a new TfeTextView and returns the TfeTextView as GtkWidget. If
an error happens, it returns \passthrough{\lstinline!NULL!}.

Returns

\begin{itemize}
\tightlist
\item
  a new TfeTextView.
\end{itemize}

\hypertarget{types-and-values-1}{%
\subsection{Types and Values}\label{types-and-values-1}}

\hypertarget{tfetextview}{%
\subsubsection{TfeTextView}\label{tfetextview}}

\begin{lstlisting}
typedef struct _TfeTextView TfeTextView
struct _TfeTextView
{
  GtkTextView parent;
  GFile *file;
};
\end{lstlisting}

The members of this structure are not allowed to be accessed by any
outer objects. If you want to obtain a copy of the GFile, use
\passthrough{\lstinline!tfe\_text\_view\_get\_file!}.

\hypertarget{tfetextviewclass}{%
\subsubsection{TfeTextViewClass}\label{tfetextviewclass}}

\begin{lstlisting}
typedef struct {
  GtkTextViewClass parent_class;
} TfeTextViewClass;
\end{lstlisting}

No member is added because TfeTextView is a final type object.

\hypertarget{enum-tfetextviewopenresponsetype}{%
\subsubsection{enum
TfeTextViewOpenResponseType}\label{enum-tfetextviewopenresponsetype}}

Predefined values for the response id given by
\passthrough{\lstinline!open-response!} signal.

Members:

\begin{itemize}
\tightlist
\item
  TFE\_OPEN\_RESPONSE\_SUCCESS: The file is successfully opened.
\item
  TFE\_OPEN\_RESPONSE\_CANCEL: Reading file is canceled by the user.
\item
  TFE\_OPEN\_RESPONSE\_ERROR: An error happened during the opening or
  reading process.
\end{itemize}

\hypertarget{signals-1}{%
\subsection{Signals}\label{signals-1}}

\hypertarget{change-file}{%
\subsubsection{change-file}\label{change-file}}

\begin{lstlisting}
void
user_function (TfeTextView *tv,
               gpointer user_data)
\end{lstlisting}

Emitted when the GFile in the TfeTextView object is changed. The signal
is emitted when:

\begin{itemize}
\tightlist
\item
  a new file is opened and read
\item
  a user choose a file with GtkFileChooserDialog and save the contents.
\item
  an error occured during I/O operation, and GFile is removed as a
  result.
\end{itemize}

\hypertarget{open-response}{%
\subsubsection{open-response}\label{open-response}}

\begin{lstlisting}
void
user_function (TfeTextView *tv,
               TfeTextViewOpenResponseType response-id,
               gpointer user_data)
\end{lstlisting}

Emitted after the user calls
\passthrough{\lstinline!tfe\_text\_view\_open!}. This signal informs the
status of file opening and reading.
