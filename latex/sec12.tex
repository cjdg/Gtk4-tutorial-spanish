\hypertarget{signals}{%
\section{Signals}\label{signals}}

\hypertarget{signals-1}{%
\subsection{Signals}\label{signals-1}}

In Gtk programming, each object is encapsulated. And it is not
recommended to use global variables because they tend to make the
program complicated. So, we need something to communicate between
objects. There are two ways to do so.

\begin{itemize}
\tightlist
\item
  Functions. For example,
  \passthrough{\lstinline!tb = gtk\_text\_view\_get\_buffer (tv)!}. The
  caller requests \passthrough{\lstinline!tv!} to give
  \passthrough{\lstinline!tb!}, which is a GtkTextBuffer instance
  connected to \passthrough{\lstinline!tv!} to the caller.
\item
  Signals. For example, \passthrough{\lstinline!activate!} signal on
  GApplication object. When the application is activated, the signal is
  emitted. Then the handler, which has been connected to the signal, is
  invoked.
\end{itemize}

The caller of the function or the handler connected to the signal is
usually out of the object. One of the difference between these two is
that the object is active or passive. In functions the object passively
responds to the caller. In signals the object actively sends a signal to
the handler.

GObject signals are registered, connected and emitted.

\begin{enumerate}
\def\labelenumi{\arabic{enumi}.}
\tightlist
\item
  Signals are registered with the object type on which they are emitted.
  The registration is done usually when the object class is initialized.
\item
  Signals are connected to handlers by
  \passthrough{\lstinline!g\_connect\_signal!} or its family functions.
  The connection is usually done out of the object.
\item
  When Signals are emitted, the connected handlers are invoked. Signal
  is emitted on the instance of the object.
\end{enumerate}

\hypertarget{signal-registration}{%
\subsection{Signal registration}\label{signal-registration}}

In TfeTextView, two signals are registered.

\begin{itemize}
\tightlist
\item
  ``change-file'' signal. This signal is emitted when
  \passthrough{\lstinline!tv->file!} is changed.
\item
  ``open-response'' signal.
  \passthrough{\lstinline!tfe\_text\_view\_open!} function is not able
  to return the status because it uses GtkFileChooserDialog. This signal
  is emitted instead of the return value of the function.
\end{itemize}

A static variable or array is used to store the signal ID. A static
array is used to register two or more signals.

\begin{lstlisting}[language=C]
enum {
  CHANGE_FILE,
  OPEN_RESPONSE,
  NUMBER_OF_SIGNALS
};

static guint tfe_text_view_signals[NUMBER_OF_SIGNALS];
\end{lstlisting}

Signals are registered in the class initialization function.

\begin{lstlisting}[language=C, numbers=left]
static void
tfe_text_view_class_init (TfeTextViewClass *class) {
  GObjectClass *object_class = G_OBJECT_CLASS (class);

  object_class->dispose = tfe_text_view_dispose;
  tfe_text_view_signals[CHANGE_FILE] = g_signal_new ("change-file",
                                 G_TYPE_FROM_CLASS (class),
                                 G_SIGNAL_RUN_LAST | G_SIGNAL_NO_RECURSE | G_SIGNAL_NO_HOOKS,
                                 0 /* class offset */,
                                 NULL /* accumulator */,
                                 NULL /* accumulator data */,
                                 NULL /* C marshaller */,
                                 G_TYPE_NONE /* return_type */,
                                 0     /* n_params */
                                 );
  tfe_text_view_signals[OPEN_RESPONSE] = g_signal_new ("open-response",
                                 G_TYPE_FROM_CLASS (class),
                                 G_SIGNAL_RUN_LAST | G_SIGNAL_NO_RECURSE | G_SIGNAL_NO_HOOKS,
                                 0 /* class offset */,
                                 NULL /* accumulator */,
                                 NULL /* accumulator data */,
                                 NULL /* C marshaller */,
                                 G_TYPE_NONE /* return_type */,
                                 1     /* n_params */,
                                 G_TYPE_INT
                                 );
}
\end{lstlisting}

\begin{itemize}
\tightlist
\item
  6-15: Registers ``change-file'' signal.
  \passthrough{\lstinline!g\_signal\_new!} function is used. The signal
  ``change-file'' has no default handler (object method handler). You
  usually don't need to set a default handler. If you need it, use
  \passthrough{\lstinline!g\_signal\_new\_class\_handler!} function. See
  \href{https://docs.gtk.org/gobject/func.signal_new_class_handler.html}{GObject
  API Reference, g\_signal\_new\_class\_handler} for further
  information.
\item
  The return value of \passthrough{\lstinline!g\_signal\_new!} is the
  signal id. The type of signal id is guint, which is the same as
  unsigned int. It is used in the function
  \passthrough{\lstinline!g\_signal\_emit!}.
\item
  16-26: Registers ``open-response'' signal. This signal has a
  parameter.
\item
  24: Number of the parameters. ``open-response'' signal has one
  parameter.
\item
  25: The type of the parameter. \passthrough{\lstinline!G\_TYPE\_INT!}
  is a type of integer. Such fundamental types are described in
  \href{https://developer-old.gnome.org/gobject/stable/gobject-Type-Information.html}{GObject
  reference manual}.
\end{itemize}

The handlers are declared as follows.

\begin{lstlisting}[language=C]
/* "change-file" signal handler */
void
user_function (TfeTextView *tv,
               gpointer user_data)

/* "open-response" signal handler */
void
user_function (TfeTextView *tv,
               TfeTextViewOpenResponseType response-id,
               gpointer user_data)
\end{lstlisting}

\begin{itemize}
\tightlist
\item
  Because ``change-file'' signal doesn't have parameter, the handler's
  parameters are a TfeTextView instance and user data.
\item
  Because ``open-response'' signal has one parameter, the handler's
  parameters are a TfeTextView instance, the signal's parameter and user
  data.
\item
  \passthrough{\lstinline!tv!} is the object instance on which the
  signal is emitted.
\item
  \passthrough{\lstinline!user\_data!} comes from the fourth argument of
  \passthrough{\lstinline!g\_signal\_connect!}.
\item
  \passthrough{\lstinline!parameter!} comes from the fourth argument of
  \passthrough{\lstinline!g\_signal\_emit!}.
\end{itemize}

The values of the parameter is defined in
\passthrough{\lstinline!tfetextview.h!} because they are public.

\begin{lstlisting}[language=C]
/* "open-response" signal response */
enum
{
  TFE_OPEN_RESPONSE_SUCCESS,
  TFE_OPEN_RESPONSE_CANCEL,
  TFE_OPEN_RESPONSE_ERROR
};
\end{lstlisting}

\begin{itemize}
\tightlist
\item
  The parameter is set to
  \passthrough{\lstinline!TFE\_OPEN\_RESPONSE\_SUCCESS!} when
  \passthrough{\lstinline!tfe\_text\_view\_open!} has successfully
  opened a file and read it.
\item
  The parameter is set to
  \passthrough{\lstinline!TFE\_OPEN\_RESPONSE\_CANCEL!} when the user
  has canceled.
\item
  The parameter is set to
  \passthrough{\lstinline!TFE\_OPEN\_RESPONSE\_ERROR!} when an error has
  occurred.
\end{itemize}

\hypertarget{signal-connection}{%
\subsection{Signal connection}\label{signal-connection}}

A signal and a handler are connected by the function
\passthrough{\lstinline!g\_signal\_connect!}. There are some similar
functions like \passthrough{\lstinline!g\_signal\_connect\_after!},
\passthrough{\lstinline!g\_signal\_connect\_swapped!} and so on.
However, \passthrough{\lstinline!g\_signal\_connect!} is the most
common. The signals ``change-file'' is connected to a callback function
out of the TfeTextView object. In the same way, the signals
``open-response'' is connected to a callback function out of the
TfeTextView object. Those callback functions are defined by users.

In the program \passthrough{\lstinline!tfe!}, callback functions are
defined in \passthrough{\lstinline!tfenotebook.c!}. And their names are
\passthrough{\lstinline!file\_changed!} and
\passthrough{\lstinline!open\_response!}. They will be explained later.

\begin{lstlisting}[language=C]
g_signal_connect (GTK_TEXT_VIEW (tv), "change-file", G_CALLBACK (file_changed), nb);

g_signal_connect (TFE_TEXT_VIEW (tv), "open-response", G_CALLBACK (open_response), nb);
\end{lstlisting}

\hypertarget{signal-emission}{%
\subsection{Signal emission}\label{signal-emission}}

Signals are emitted on an instance. The type of the instance is the
second argument of \passthrough{\lstinline!g\_signal\_new!}. The
relationship between the signal and object type is determined when the
signal is registered.

A function \passthrough{\lstinline!g\_signal\_emit!} is used to emit the
signal. The following lines are extracted from
\passthrough{\lstinline!tfetextview.c!}. Each line comes from a
different line.

\begin{lstlisting}[language=C]
g_signal_emit (tv, tfe_text_view_signals[CHANGE_FILE], 0);
g_signal_emit (tv, tfe_text_view_signals[OPEN_RESPONSE], 0, TFE_OPEN_RESPONSE_SUCCESS);
g_signal_emit (tv, tfe_text_view_signals[OPEN_RESPONSE], 0, TFE_OPEN_RESPONSE_CANCEL);
g_signal_emit (tv, tfe_text_view_signals[OPEN_RESPONSE], 0, TFE_OPEN_RESPONSE_ERROR);
\end{lstlisting}

\begin{itemize}
\tightlist
\item
  The first argument is the instance on which the signal is emitted.
\item
  The second argument is the signal id.
\item
  The third argument is the detail of the signal. ``change-file'' signal
  and ``open-response'' signal doesn't have details and the argument is
  zero when no details.
\item
  ``change-file'' signal doesn't have parameter, so there's no fourth
  parameter.
\item
  ``open-response'' signal has one parameter. The fourth parameter is
  the parameter.
\end{itemize}
