\hypertarget{prerequisite-and-license}{%
\section{Prerequisite and License}\label{prerequisite-and-license}}

\hypertarget{prerequisite}{%
\subsection{Prerequisite}\label{prerequisite}}

\hypertarget{gtk4-on-a-linux-os}{%
\subsubsection{Gtk4 on a Linux OS}\label{gtk4-on-a-linux-os}}

This tutorial is about Gtk4 libraries. It is originally used on Linux
with C compiler, but now it is used more widely, on Windows and MacOS,
with Vala, Python and so on. However, this tutorial describes only
\emph{C programs on Linux}.

If you want to try the examples in the tutorial, you need:

\begin{itemize}
\tightlist
\item
  PC with Linux distribution like Ubuntu, Debian and so on.
\item
  Gcc.
\item
  Gtk4. The stable version of Gtk on Linux distributions is version
  three at present. You need to install Gtk4 to your computer. See
  Section 3 for the installation of Gtk4.
\end{itemize}

\hypertarget{ruby-and-rake-for-making-the-document}{%
\subsubsection{Ruby and rake for making the
document}\label{ruby-and-rake-for-making-the-document}}

This repository includes Ruby programs. They are used to make Markdown
files, HTML files, Latex files and a PDF file.

You need:

\begin{itemize}
\tightlist
\item
  Linux distribution like Ubuntu.
\item
  Ruby programming language. There are two ways to install it. One is
  installing the distribution's package. The other is using rbenv and
  ruby-build. If you want to use the latest version of ruby, use rbenv.
\item
  Rake. It is a gem, which is a library written in Ruby. You can install
  it as a package of your distribution or use gem command.
\end{itemize}

\hypertarget{license}{%
\subsection{License}\label{license}}

Copyright (C) 2020 ToshioCP (Toshio Sekiya)

Gtk4 tutorial repository contains the tutorial document and software
such as converters, generators and controllers. All of them make up the
`Gtk4 tutorial' package. This package is simply called `Gtk4 tutorial'
in the following description. `Gtk4 tutorial' is free; you can
redistribute it and/or modify it under the terms of the GNU General
Public License as published by the Free Software Foundation; either
version 3 of the License or, at your option, any later version.

`Gtk4 tutorial' is distributed in the hope that it will be useful, but
WITHOUT ANY WARRANTY; without even the implied warranty of
MERCHANTABILITY or FITNESS FOR A PARTICULAR PURPOSE. See the
\href{https://www.gnu.org/licenses/gpl-3.0.html}{GNU General Public
License} for more details.
