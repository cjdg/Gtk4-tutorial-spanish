\hypertarget{prerequisitos-y-licencia}{%
\section{Prerequisitos y licencia}\label{prerequisitos-y-licencia}}

\hypertarget{prerequisitos}{%
\subsection{Prerequisitos}\label{prerequisitos}}

\hypertarget{gtk4-en-linux}{%
\subsubsection{Gtk4 en Linux}\label{gtk4-en-linux}}

Este tutorial es acerca de las librerías Gtk4. Se usan en Linux con un
compilador C, pero ahora se usan mas ampliamente, en Windows y MacOs,
con Vala, Python, etc. Sin embargo, este tutorial trata sólo
\emph{programas Linux escritos en C}.

Si deseas probar los ejemplos de este tutorial, necesitas:

\begin{itemize}
\tightlist
\item
  PC con Linux cómo Ubuntu, Debian, etc.
\item
  Gcc.
\item
  Gtk4.
\end{itemize}

La versión estable actual en las distribuciones recientes de Linux es la
4. Necesitas instalar la librería en tu computadora. Ve Sección 3 para
la instalación de Gtk4.

\hypertarget{ruby-y-rake-para-crear-la-documentaciuxf3n}{%
\subsubsection{Ruby y rake para crear la
documentación}\label{ruby-y-rake-para-crear-la-documentaciuxf3n}}

Este repositorio incluye programas escritos en Ruby. Se usan para hacer
los archivos Markdown, HTML, Latex y PDF.

Necesitas:

\begin{itemize}
\tightlist
\item
  Distribución Linux como Ubuntu.
\item
  Lengua Ruby instalado.
\end{itemize}

Hay 2 maneras de instalarlo. Uno es usar los paquetes de la
distribución. La otra es usando rbenv y ruby-build. Si deseas usar la
última version de Ruby, usa rbenv. - Rake Es una gema, que es una
líbrería escrita en Ruby. Puedes instalarlo como un paquete en la
distribución o usar el comando gem.

\hypertarget{licencia}{%
\subsection{Licencia}\label{licencia}}

Copyright (C) 2020 ToshioCP (Toshio Sekiya)

Gtk4 tutorial repository contains the tutorial document and software
such as converters, generators and controllers. All of them make up the
`Gtk4 tutorial' package. This package is simply called `Gtk4 tutorial'
in the following description. `Gtk4 tutorial' is free; you can
redistribute it and/or modify it under the terms of the GNU General
Public License as published by the Free Software Foundation; either
version 3 of the License or, at your option, any later version.

`Gtk4 tutorial' is distributed in the hope that it will be useful, but
WITHOUT ANY WARRANTY; without even the implied warranty of
MERCHANTABILITY or FITNESS FOR A PARTICULAR PURPOSE. See the
\href{https://www.gnu.org/licenses/gpl-3.0.html}{GNU General Public
License} for more details.
