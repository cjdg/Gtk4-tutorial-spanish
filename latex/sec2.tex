\hypertarget{installing-gtk4-into-linux-distributions}{%
\section{Installing Gtk4 into Linux
distributions}\label{installing-gtk4-into-linux-distributions}}

This section describes how to install Gtk4 into Linux distributions.

This tutorial is without any warranty. If you want to install Gtk4 to
your computer, do it at your own risk.

The information in this section is the one on April/27/2022. The words
`at present' and/or `now' in this section means `April/27/2022'.

There are three possible way to install Gtk4.

\begin{itemize}
\tightlist
\item
  Install it from the distribution packages.
\item
  Build it from the source file.
\item
  Install a Gnome 40 distribution with the gnome-boxes.
\end{itemize}

\hypertarget{installation-from-the-distribution-packages}{%
\subsection{Installation from the distribution
packages}\label{installation-from-the-distribution-packages}}

The first way is easy to install. It is a recommended way. I've
installed Gtk4 packages in Ubuntu 21.04. (Now, my Ubuntu version is
21.10).

\begin{lstlisting}
$ sudo apt-get install libgtk-4-bin libgtk-4-common libgtk-4-dev libgtk-4-doc
\end{lstlisting}

Fedora, Arch, Debian and OpenSUSE are also possible. See
\href{https://www.gtk.org/docs/installations/linux\#installing-gtk-from-packages}{Installing
GTK from packages}. The following table shows the distributions which
support Gtk4.

\begin{longtable}[]{@{}cccc@{}}
\toprule
Distribution & version & Gtk4 & Gnome40\tabularnewline
\midrule
\endhead
Fedora & 36 & 4.4.2 & Gnome42\tabularnewline
Ubuntu & 22.04lts & 4.4 & Gnome41(4.6.2)\tabularnewline
Debian & bookworm(testing) & 4.6.5 & Gnome42\tabularnewline
Arch & rolling release & 4.6.5 & Gnome42\tabularnewline
Gentoo & rolling release & 4.6.5 & Gnome42\tabularnewline
OpenSUSE & Tumbleweed(rolling release) & 4.6.5 & Gnome42\tabularnewline
\bottomrule
\end{longtable}

If you've installed Gtk4 from the packages, you don't need to read the
rest of this section.

\hypertarget{installation-from-the-source-file}{%
\subsection{Installation from the source
file}\label{installation-from-the-source-file}}

If your operating system doesn't have Gtk4 packages, you need to build
it from the source. Or, if you want the latest version of Gtk4 (4.6.3),
you also need to build it from the source.

I installed Gtk4 from the source in January 2021. So, the following
information is old, especially for the version of each software. For the
latest information, see
\href{https://docs.gtk.org/gtk4/building.html}{Gtk API Reference,
Building GTK}.

\hypertarget{prerequisites-for-gtk4-installation}{%
\subsubsection{Prerequisites for Gtk4
installation}\label{prerequisites-for-gtk4-installation}}

\begin{itemize}
\tightlist
\item
  Linux operating system. For example, Ubuntu 20.10 or 20.04LTS. Other
  distributions might be OK.
\item
  Packages for development such as gcc, meson, ninja, git, wget and so
  on.
\item
  Dev package is necessary for each software below.
\end{itemize}

\hypertarget{installation-target}{%
\subsubsection{Installation target}\label{installation-target}}

I installed Gtk4 under the directory
\passthrough{\lstinline!$HOME/local!}. This is a private user area.

If you want to install it in the system area,
\passthrough{\lstinline!/opt/gtk4!} is one of good choices.
\href{https://docs.gtk.org/gtk4/building.html}{Gtk API Reference,
Building GTK} gives an installation example to
\passthrough{\lstinline!/opt/gtk4!}.

Don't install it to \passthrough{\lstinline!/usr/local!} which is the
default. It is used by Ubuntu applications which are not build on Gtk4.
Therefore, the risk is high and probably bad things will happen.
Actually I did it and I needed to reinstall Ubuntu.

\hypertarget{installation-to-ubuntu-20.10}{%
\subsubsection{Installation to Ubuntu
20.10}\label{installation-to-ubuntu-20.10}}

Most of the necessary libraries are included by Ubuntu 20.10. Therefore,
they can be installed with \passthrough{\lstinline!apt-get!} command.
You don't need to install them from the source tarballs. You can skip
the subsections below about prerequisite library installation (Glib,
Pango, Gdk-pixbuf and Gtk-doc).

\hypertarget{glib-installation}{%
\subsubsection{Glib installation}\label{glib-installation}}

If your Ubuntu is 20.04LTS, you need to install prerequisite libraries
from the tarballs. Check the version of your library and if it is lower
than the necessary version, install it from the source.

For example,

\begin{lstlisting}
$ pkg-config --modversion glib-2.0
2.64.6
\end{lstlisting}

The necessary version is 2.66.0 or higher. Therefore, the example above
shows that you need to install Glib.

I installed 2.67.1 which was the latest version at that time (January
2021). Download Glib source files from the repository, then decompress
and extract files.

\begin{lstlisting}
$ wget https://download.gnome.org/sources/glib/2.67/glib-2.67.1.tar.xz
$ tar -Jxf glib-2.67.1.tar.xz
\end{lstlisting}

Some packages are required to build Glib. You can find them if you run
meson.

\begin{lstlisting}
$ meson --prefix $HOME/local _build
\end{lstlisting}

Use apt-get and install the prerequisites. For example,

\begin{lstlisting}
$ sudo apt-get install -y  libpcre2-dev libffi-dev
\end{lstlisting}

After that, compile Glib.

\begin{lstlisting}
$ rm -rf _build
$ meson --prefix $HOME/local _build
$ ninja -C _build
$ ninja -C _build install
\end{lstlisting}

Set several environment variables so that the Glib libraries installed
can be used by build tools. Make a text file below and save it as
\passthrough{\lstinline!env.sh!}

\begin{lstlisting}
# compiler
CPPFLAGS="-I$HOME/local/include"
LDFLAGS="-L$HOME/local/lib"
PKG_CONFIG_PATH="$HOME/local/lib/pkgconfig:$HOME/local/lib/x86_64-linux-gnu/pkgconfig"
export CPPFLAGS LDFLAGS PKG_CONFIG_PATH
# linker
LD_LIBRARY_PATH="$HOME/local/lib/x86_64-linux-gnu/"
PATH="$HOME/local/bin:$PATH"
export LD_LIBRARY_PATH PATH
# gsetting
export GSETTINGS_SCHEMA_DIR=$HOME/local/share/glib-2.0/schemas
\end{lstlisting}

Then, use . (dot) or source command to include these commands to the
current bash.

\begin{lstlisting}
$ . env.sh
\end{lstlisting}

or

\begin{lstlisting}
$ source env.sh
\end{lstlisting}

This command carries out the commands in
\passthrough{\lstinline!env.sh!} and changes the environment variables
above in the current shell.

\hypertarget{pango-installation}{%
\subsubsection{Pango installation}\label{pango-installation}}

Download and untar.

\begin{lstlisting}
$ wget https://download.gnome.org/sources/pango/1.48/pango-1.48.0.tar.xz
$ tar -Jxf pango-1.48.0.tar.xz
\end{lstlisting}

Try meson and check the required packages. Install all the
prerequisites. Then, compile and install Pango.

\begin{lstlisting}
$ meson --prefix $HOME/local _build
$ ninja -C _build
$ ninja -C _build install
\end{lstlisting}

It installs Pango-1.0.gir under
\passthrough{\lstinline!$HOME/local/share/gir-1.0!}. If you installed
Pango without \passthrough{\lstinline!--prefix!} option, then it would
be located at \passthrough{\lstinline!/usr/local/share/gir-1.0!}. This
directory (/usr/local/share) is used by applications. They find the
directory by the environment variable
\passthrough{\lstinline!XDG\_DATA\_DIRS!}. It is a text file which keep
the list of `share' directories like
\passthrough{\lstinline!/usr/share!},
\passthrough{\lstinline!usr/local/share!} and so on. Now
\passthrough{\lstinline!$HOME/local/share!} needs to be added to
\passthrough{\lstinline!XDG\_DATA\_DIRS!}, or error will occur in the
later compilation.

\begin{lstlisting}
$ export XDG_DATA_DIRS=$HOME/local/share:$XDG_DATA_DIRS
\end{lstlisting}

\hypertarget{gdk-pixbuf-and-gtk-doc-installation}{%
\subsubsection{Gdk-pixbuf and Gtk-doc
installation}\label{gdk-pixbuf-and-gtk-doc-installation}}

Download and untar.

\begin{lstlisting}
$ wget https://download.gnome.org/sources/gdk-pixbuf/2.42/gdk-pixbuf-2.42.2.tar.xz
$ tar -Jxf gdk-pixbuf-2.42.2.tar.xz
$ wget https://download.gnome.org/sources/gtk-doc/1.33/gtk-doc-1.33.1.tar.xz
$ tar -Jxf gtk-doc-1.33.1.tar.xz
\end{lstlisting}

Same as before, install prerequisite packages, then compile and install
them.

The installation of Gtk-doc put \passthrough{\lstinline!gtk-doc.pc!}
under \passthrough{\lstinline!$HOME/local/share/pkgconfig!}. This file
is used by pkg-config, which is one of the build tools. The directory
needs to be added to the environment variable
\passthrough{\lstinline!PKG\_CONFIG\_PATH!}

\begin{lstlisting}
$ export PKG_CONFIG_PATH="$HOME/local/share/pkgconfig:$PKG_CONFIG_PATH"
\end{lstlisting}

\hypertarget{gtk4-installation}{%
\subsubsection{Gtk4 installation}\label{gtk4-installation}}

If you want the latest development version of Gtk4, use git and clone
the repository.

\begin{lstlisting}
$ git clone https://gitlab.gnome.org/GNOME/gtk.git
\end{lstlisting}

If you want a stable version of Gtk4, then download it from
\href{https://download.gnome.org/sources/gtk/}{Gnome source website}.
The latest version is 4.3.1 (13/June/2021).

Compile and install it.

\begin{lstlisting}
$ meson --prefix $HOME/local _build
$ ninja -C _build
$ ninja -C _build install
\end{lstlisting}

If you want to know more information, refer to
\href{https://docs.gtk.org/gtk4/building.html}{Gtk4 API Reference,
Building GTK}.

\hypertarget{modify-env.sh}{%
\subsubsection{Modify env.sh}\label{modify-env.sh}}

Because environment variables disappear when you log out, you need to
add them again. Modify \passthrough{\lstinline!env.sh!}.

\begin{lstlisting}
# compiler
CPPFLAGS="-I$HOME/local/include"
LDFLAGS="-L$HOME/local/lib"
PKG_CONFIG_PATH="$HOME/local/lib/pkgconfig:$HOME/local/lib/x86_64-linux-gnu/pkgconfig:
$HOME/local/share/pkgconfig"
export CPPFLAGS LDFLAGS PKG_CONFIG_PATH
# linker
LD_LIBRARY_PATH="$HOME/local/lib/x86_64-linux-gnu/"
PATH="$HOME/local/bin:$PATH"
export LD_LIBRARY_PATH PATH
# gir
XDG_DATA_DIRS=$HOME/local/share:$XDG_DATA_DIRS
export XDG_DATA_DIRS
# gsetting
export GSETTINGS_SCHEMA_DIR=$HOME/local/share/glib-2.0/schemas
# girepository-1.0
export GI_TYPELIB_PATH=$HOME/local/lib/x86_64-linux-gnu/girepository-1.0
\end{lstlisting}

Include this file by . (dot) command before using Gtk4 libraries.

You may think you can add them in your
\passthrough{\lstinline!.profile!}. But it's a wrong decision. Never
write them to your \passthrough{\lstinline!.profile!}. The environment
variables above are necessary only when you compile and run Gtk4
applications. Otherwise it's not necessary. If you changed the
environment variables above and run Gtk3 applications, it probably
causes serious damage.

\hypertarget{compiling-gtk4-applications}{%
\subsubsection{Compiling Gtk4
applications}\label{compiling-gtk4-applications}}

Before you compile Gtk4 applications, define environment variables
above.

\begin{lstlisting}
$ . env.sh
\end{lstlisting}

After that you can compile them without anything. For example, to
compile \passthrough{\lstinline!sample.c!}, type the following.

\begin{lstlisting}
$ gcc `pkg-config --cflags gtk4` sample.c `pkg-config --libs gtk4`
\end{lstlisting}

To know how to compile Gtk4 applications, refer to the section 3
(GtkApplication and GtkApplicationWindow) and after.

\hypertarget{installing-fedora-34-with-gnome-boxes}{%
\subsection{Installing Fedora 34 with
gnome-boxes}\label{installing-fedora-34-with-gnome-boxes}}

The last part of this section is about Gnome40 and gnome-boxes. Gnome 40
is a new version of Gnome desktop system. And Gtk4 is installed in the
distribution. See \href{https://forty.gnome.org/}{Gnome 40 website}
first.

\emph{However, Gnome40 is not necessary to compile and run Gtk4
applications.}

There are seven choices at present.

\begin{itemize}
\tightlist
\item
  Gnome OS
\item
  Arch Linux
\item
  Gentoo Linux
\item
  Fedora 36
\item
  openSUSE Tumbleweed
\item
  Ubuntu 22.04
\item
  Debian bookworm
\end{itemize}

I've installed Fedora 34 with gnome-boxes. My OS was Ubuntu 21.04 at
that time. Gnome-boxes creates a virtual machine in Ubuntu and Fedora
will be installed to that virtual machine.

The instruction is as follows.

\begin{enumerate}
\def\labelenumi{\arabic{enumi}.}
\tightlist
\item
  Download Fedora 34 iso file. There is an link at the end of
  \href{https://forty.gnome.org/}{Gnome 40 website}.
\item
  Install gnome-boxes with apt-get command.
\end{enumerate}

\begin{lstlisting}
$ sudo apt-get install gnome-boxes
\end{lstlisting}

\begin{enumerate}
\def\labelenumi{\arabic{enumi}.}
\setcounter{enumi}{2}
\tightlist
\item
  Run gnome-boxes.
\item
  Click on \passthrough{\lstinline!+!} button on the top left corner and
  launch a box creation wizard by clicking
  \passthrough{\lstinline!Create a Virtual Machine ...!}. Then a dialog
  appears. Click on
  \passthrough{\lstinline!Operationg System Image File!} and select the
  iso file you have downloaded.
\item
  Then, the Fedora's installer is executed. Follow the instructions by
  the installer. At the end of the installation, the installer instructs
  to reboot the system. Click on the right of the title bar and select
  reboot or shutdown.
\item
  Your display is back to the initial window of gnome-boxes, but there
  is a button \passthrough{\lstinline!Fedora 34 Workstation!} on the
  upper left of the window. Click on the button then Fedora will be
  executed.
\item
  A setup dialog appears. Setup Fedora according to the wizard.
\end{enumerate}

Now you can use Fedora. It includes Gtk4 libraries already. But you need
to install the Gtk4 development package. Use
\passthrough{\lstinline!dnf!} to install
\passthrough{\lstinline!gtk4.x86\_64!} package.

\begin{lstlisting}
$ sudo dnf install gtk4.x86_64
\end{lstlisting}

\hypertarget{gtk4-compilation-test}{%
\subsubsection{Gtk4 compilation test}\label{gtk4-compilation-test}}

You can test the Gtk4 development packages by compiling files which are
based on Gtk4. I've tried compiling \passthrough{\lstinline!tfe!} text
editor, which is written in section 21.

\begin{enumerate}
\def\labelenumi{\arabic{enumi}.}
\tightlist
\item
  Run Firefox.
\item
  Open this website
  (\href{https://github.com/ToshioCP/Gtk4-tutorial}{Gtk4-Tutorial}).
\item
  Click on the green button labeled \passthrough{\lstinline!Code!}.
\item
  Select \passthrough{\lstinline!Download ZIP!} and download the codes
  from the repository.
\item
  Unzip the file.
\item
  Change your current directory to \passthrough{\lstinline!src/tfe7!}.
\item
  Compile it.
\end{enumerate}

\begin{lstlisting}
$ meson _build
bash: meson: command not found...
Install package 'meson' to provide command 'meson'? [N/y] y

 * Waiting in queue...
The following packages have to be installed:
 meson-0.56.2-2.fc34.noarch    High productivity build system
 ninja-build-1.10.2-2.fc34.x86_64    Small build system with a focus on speed
 vim-filesystem-2:8.2.2787-1.fc34.noarch    VIM filesystem layout
Proceed with changes? [N/y] y

... ...
... ...

The Meson build system
Version: 0.56.2

... ...
... ...

Project name: tfe
Project version: undefined
C compiler for the host machine: cc (gcc 11.0.0 "cc (GCC) 11.0.0 20210210 (Red Hat 11.0.0-0)")
C linker for the host machine: cc ld.bfd 2.35.1-38
Host machine cpu family: x86_64
Host machine cpu: x86_64
Found pkg-config: /usr/bin/pkg-config (1.7.3)
Run-time dependency gtk4 found: YES 4.2.0
Found pkg-config: /usr/bin/pkg-config (1.7.3)
Program glib-compile-resources found: YES (/usr/bin/glib-compile-resources)
Program glib-compile-schemas found: YES (/usr/bin/glib-compile-schemas)
Program glib-compile-schemas found: YES (/usr/bin/glib-compile-schemas)
Build targets in project: 4

Found ninja-1.10.2 at /usr/bin/ninja

$ ninja -C _build
ninja: Entering directory `_build'
[12/12] Linking target tfe

$ ninja -C _build install
ninja: Entering directory `_build'
[0/1] Installing files.
Installing tfe to /usr/local/bin
Installation failed due to insufficient permissions.
Attempting to use polkit to gain elevated privileges...
Installing tfe to /usr/local/bin
Installing /home/<username>/Gtk4-tutorial-main/src/tfe7/com.github.ToshioCP.tfe.gschema.xml to /usr/local/share/glib-2.0/schemas
Running custom install script '/usr/bin/glib-compile-schemas /usr/local/share/glib-2.0/schemas/'
\end{lstlisting}

\begin{enumerate}
\def\labelenumi{\arabic{enumi}.}
\setcounter{enumi}{7}
\tightlist
\item
  Execute it.
\end{enumerate}

\begin{lstlisting}
$ tfe
\end{lstlisting}

Then, the window of \passthrough{\lstinline!tfe!} text editor appears.
The compilation and execution have succeeded.
